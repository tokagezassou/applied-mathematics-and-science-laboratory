\documentclass[uplatex,a4j]{jsarticle}
\usepackage[dvipdfmx]{graphicx}
\usepackage{amsmath,ascmac}
\usepackage{bm}
\usepackage{algorithm,algorithmic}
\usepackage{listings}
\usepackage{subcaption}
\usepackage{amsfonts}
\usepackage{amssymb}
\lstset{%
  language={C},
  basicstyle={\small},%
  identifierstyle={\small},%
  commentstyle={\small\itshape},%
  keywordstyle={\small\bfseries},%
  ndkeywordstyle={\small},%
  stringstyle={\small\ttfamily},
  frame={tb},
  breaklines=true,
  columns=[l]{fullflexible},%
  numbers=left,%
  xrightmargin=0zw,%
  xleftmargin=3zw,%
  numberstyle={\scriptsize},%
  stepnumber=1,
  numbersep=1zw,%
  lineskip=-0.5ex%
}
\renewcommand{\lstlistingname}{コード}
\renewcommand{\lstlistlistingname}{コード目次}
\renewcommand{\thesubsection}{(\arabic{subsection})}
\renewcommand{\thesection}{\arabic{section}}



%
% 以後をレポート課題ごとに書き換える。
%

\begin{document}
\title{サンプル\LaTeX}
\author{名前(学籍番号 xxxx)}
\date{2025年 x月 y日}
\maketitle

\section{序}
\label{sec:intro}

このサンプル\LaTeX ファイルでは、\ref{sec:eqn}章では数式につい
て、\ref{sec:table}章では表について、\ref{sec:figure}章では図につい
て、\ref{sec:code}章ではソースコードの出力の仕方について、サンプル例を示した。
また、レポートの書き方の勉強には、文献\cite{Kinoshita}の一読を勧める。

% 式
\section{数式とは?}
\label{sec:eqn}

20世紀の物理学の有名な公式\eqref{eq:ein}と\eqref{eq:sch}について考える。

% &の位置でそろえる
\begin{align}
  \label{eq:ein}
   & E = mc^2                                                               \\
  \label{eq:sch}
   & i\hbar\frac{\partial \vert\Psi\rangle}{\partial t} = H\vert\Psi\rangle
\end{align}

% 表
\section{表とは?}
\label{sec:table}

表\ref{tab:num}に集計結果を示す。

\begin{table}[ht]
  \centering
  \caption{性別集計}
  \label{tab:num}
  \begin{tabular}[ht]{|c|c|}
    \hline
    男性 & 10人 \\
    \hline
    女性 & 10人 \\
    \hline
  \end{tabular}
\end{table}

% 図
\section{図とは?}
\label{sec:figure}

図\ref{fig:tn}はどおーだよ。

\begin{figure}[ht]
  \centering
  \includegraphics[width=0.8\textwidth]{clodsire.jpg}
  \caption{どおー}
  \label{fig:tn}
\end{figure}

\begin{figure}[htbp]
  \centering

  \begin{subfigure}{0.45\textwidth}
    \centering
    \includegraphics[width=\linewidth]{question6/q6_relative_error.png}
    \caption{全体}
    \label{fig:q6_re_whole}
  \end{subfigure}
  \hfill
  \begin{subfigure}{0.45\textwidth}
    \centering
    \includegraphics[width=\linewidth]{question6/q6_relative_error_zoom.png}
    \caption{中央値付近}
    \label{fig:q6_re_zoom}
  \end{subfigure}
  
  \caption{相対誤差の最大値}
  \label{fig:q6_re}
\end{figure}

\begin{figure}[htbp]
  \centering

  \begin{minipage}[t]{0.48\textwidth}
    \centering
    \includegraphics[width=\linewidth]{question4/q4_eigenvalue_relative_error.png}
    \captionof{figure}{第1固有値の相対誤差}
    \label{fig:q4_val_re}
  \end{minipage}
  \hfill
  \begin{minipage}[t]{0.48\textwidth}
    \centering
    \includegraphics[width=\linewidth]{question4/q4_eigenvector_relative_error.png}
    \captionof{figure}{第1固有ベクトルの相対誤差}
    \label{fig:q4_vec_re}
  \end{minipage}
  
\end{figure}

% コード
\section{コードとは?}
\label{sec:code}

コード\ref{code:sample}は有名なサンプルコードです。

\lstinputlisting[caption=ハローワールド,label=code:sample]{sample.go}
\lstinputlisting[
  caption=$f(x)$を出力する関数,
  label=code:1,
  linerange={1-5}
]{question18/question18.jl}

\begin{itembox}[l]{二分法}
  \begin{description}
    \item[step 0:]$f(a_0)<0, f(b_0) \geq 0$を満たす初期点$a_0, b_0$を選び, 
    終了条件$\epsilon > 0$を決める. 
    \item[step 1:] 中間点 $c_n := (a_n + b_n)/2$ を求める. 
    求めた$c_n$が終了条件$|f(c_N)| \leq \epsilon$を満たせば, $c_n$を零点として終了する. 
    \item[step 2:]$f(c_n) < 0$であれば$a_{n+1} = c_n,\ b_{b+1} = b_n$とし, 
    $f(c_n) \geq 0$であれば$a_{n+1} = a_n,\ b_{b+1} = c_n$として, step 1へ戻る. 
  \end{description}
\end{itembox}

\section{まとめ}
読みやすいレポートを書きましょう。

\begin{thebibliography}{99}
  \bibitem{Kinoshita}
  木下 是雄, 中公新書「理科系の作文技術」, 1981.
\end{thebibliography}
\end{document}
%%% Local Variables:
%%% mode: latex
%%% TeX-master: t
%%% End:
