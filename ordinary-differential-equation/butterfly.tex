\documentclass[uplatex,a4j]{jsarticle}
\usepackage[dvipdfmx]{graphicx}
\usepackage{amsmath,ascmac}
\usepackage{bm}
\usepackage{algorithm,algorithmic}
\usepackage{listings}
\usepackage{subcaption}
\usepackage{amsfonts}
\usepackage{amssymb}
\usepackage{amsmath}
\lstset{%
  language={C},
  basicstyle={\small},%
  identifierstyle={\small},%
  commentstyle={\small\itshape},%
  keywordstyle={\small\bfseries},%
  ndkeywordstyle={\small},%
  stringstyle={\small\ttfamily},
  frame={tb},
  breaklines=true,
  columns=[l]{fullflexible},%
  numbers=left,%
  xrightmargin=0zw,%
  xleftmargin=3zw,%
  numberstyle={\scriptsize},%
  stepnumber=1,
  numbersep=1zw,%
  lineskip=-0.5ex%
}
\renewcommand{\lstlistingname}{コード}
\renewcommand{\lstlistlistingname}{コード目次}
\renewcommand{\thesubsection}{(\arabic{subsection})}
\renewcommand{\thesection}{\arabic{section}}

\begin{document}

\begin{titlepage}
    \centering
    \vfill
    
    {\Huge 数理工学実験\par}
    \vspace{1cm}
    
    {\Large 常微分方程式レポート\par}
    
    \vfill
    
    {\Large
    所属: 工学部情報学科数理工学コース2年\par
    学籍番号: 1029-36-1263\par
    氏名: 天野 塁\par
    }
    
    \vfill
    
    {\Large 提出日: \today \par}
    
    \vfill
\end{titlepage}

\section{課題1 身の回りの常微分方程式}
\label{sec:q1}

上空から落下する雨粒の運動を考える. 雨粒の位置を$x$, 速度を$v$, 質量を$m$とし, 重力加速度を$g$とする. 
また落下する雨粒に働く空気抵抗は速度に比例し, $-kv \ (k > 0)$と表されるとする. 
また初期位置及び初期速度を$x_0, v_0$とする.
このとき雨粒の運動は 
\begin{align}
  \begin{cases}
    x' = v \\
    m v' = mg - kv \\
    x(0) = x_0 \\
    v(0) = v_0
  \end{cases}
\end{align}
で表される. 


\section{課題2 常微分方程式の数値計算法の構成}
\label{sec:q2}
4次のAdams-Bashforth法及び4次のAdams-Moulton法は式\eqref{eq:ab}, \eqref{eq:am}の通り. 
ただし特に引数を明記しない限り$f_k = f_k(t_k, u_k)$とする. 
\begin{align}
  u_n &\approx u_{n-1} + \frac{\Delta t}{24} (55f_{n-1} - 59f_{n-2} + 37f_{n-3} - 9f_{n-4})
  \label{eq:ab} \\
  u_n &\approx u_{n-1} + \frac{\Delta t}{24} (9f_n + 19f_{n-1} - 5f_{n-2} + f_{n-3})
  \label{eq:am}
\end{align}
また式\eqref{eq:ab}, \eqref{eq:am}を用いた予測子・修正子法は式\eqref{eq:q2}の通り. 

\begin{align}
  \begin{cases}
    u_n^* &\approx u_{n-1} + \frac{\Delta t}{24} (55f_{n-1} - 59f_{n-2} + 37f_{n-3} - 9f_{n-4}) \\
    u_n &\approx u_{n-1} + \frac{\Delta t}{24} (9f(t_n, u_n^*) + 19f_{n-1} - 5f_{n-2} + f_{n-3})
  \end{cases}
  \label{eq:q2}
\end{align}


\section{課題3 常微分方程式の計算法の精度}
\label{sec:q3}
常微分方程式の初期値問題
\begin{gather}
  u' = u, \ u(0) = 1
\end{gather}
を$t \in [0,1]$の範囲で数値的に解いて精度を比較する. 
ステップ数を$N = 2^i$とし, $i$の値を$i = 1, 2, ... , 8$と変化させた. 
このとき$t = 1$における数値解$u_N^{i}$と解析解の差$E^{(i)}$は
\begin{gather}
  E^{(i)} = |u_N^{i} - e|
\end{gather}
と表され, これを用いて計算法の次数$p$は
\begin{gather}
  p \approx \log_2{\frac{E^{(i - 1)}}{E^{(i)}}}
\end{gather}
と表され, 右辺の$i$が大きくなるほど$p$の値は次数に近づく. \\

前進Euler法, 2次及び3次のAdams-Bashforth法, Heun法, 4次のRunge-Kutta法の
計算次数の漸近の様子は図\ref{fig:q3}の通りで, 最終的な計算次数は表\ref{tab:q3}の通り.

\begin{figure}[ht]
  \centering
  \includegraphics[width=0.8\textwidth]{question3/q3.png}
  \caption{計算法の次数}
  \label{fig:q3}
\end{figure}

\begin{table}[ht]
  \centering
  \caption{計算法の次数}
  \label{tab:q3}
  \begin{tabular}[ht]{c c}
    \hline
    前進Euler法 & 1次 \\
    2次のAdams-Bashforth法 & 2次 \\
    3次のAdams-Bashforth法 & 3次 \\
    Heun法 & 2次 \\
    4次のRunge-Kutta法 & 4次 \\
    \hline
  \end{tabular}
\end{table}

\section{課題4 常微分方程式の計算法の安定性}
\label{sec:q4}
$t \in [0, \infty]$における常微分方程式の初期値問題
\begin{gather}
  u' = -2u + 1, \ u(0) = 1
\end{gather}
を考える. \\
このときCrank-Nicolson法は
\begin{gather}
  u_n \approx u_{n-1} + \frac{\Delta t}{2} [(-2u_{n-1} + 1) + (-2u_n + 1)]
\end{gather}
で表され, $u_n \approx a_1 u_{n^1} + a_0$の形に変形すると, 
\begin{gather}
  a_1 = \frac{1 - \Delta t}{1 + \Delta t}
\end{gather}
となる. ここで$\Delta t \geq 0$より, 常に$|a_1| \leq 0$である. 
つまりCrank-Nicolson法は$\Delta t$の値によらず常に安定である. \\
一方1次のAdams-Bashforth法と2次のAdams-Moulton法を組み合わせた予測子・修正子法, すなわちHeun法は
\begin{align}
  u_n &\approx u_{n-1} + \frac{\Delta t}{2} {-2(u_{n-1} - 2\Delta t u_{n-1} - 2 u_{n-1})} + const \\
  &= (\frac{1}{2}x^2 - x + 1)u_{n-1} + const \ \text{(ただし$x = 2\Delta t$)}
\end{align}
で表され,
\begin{gather}
  a_1 = \frac{1}{2}x^2 - x + 1
\end{gather}
となる. ここで$|a_1| \leq 1$を解くと$0 \leq x \leq 2$となるので, Heun法が安定する$\Delta t$の範囲は
\begin{gather}
  0 \leq \Delta t \leq 1
\end{gather}
である. \\
実際の数値計算は$\Delta t = 0.5, 1.5$の2種類の値で$t \in [0, 50]$の範囲で実行した. 
$\Delta t = 0.5$及び$\Delta t = 1.5$のときの解の挙動は, それぞれ\ref{fig:q4-1}, \ref{fig:q4-2}の通り. 
いずれの場合でもCrank-Nicolson法で求めた数値解は解析解と同じ挙動で収束している. 
一方Heun法で求めた数値解は, $\Delta t = 1.5$の場合に指数関数のオーダーで発散していることが確認できる. 

\begin{figure}[htbp]
  \centering

  \begin{subfigure}{0.45\textwidth}
    \centering
    \includegraphics[width=\linewidth]{question4/q4-1.png}
    \caption{$\Delta t = 0.5$}
    \label{fig:q4-1}
  \end{subfigure}
  \hfill
  \begin{subfigure}{0.45\textwidth}
    \centering
    \includegraphics[width=\linewidth]{question4/q4-2.png}
    \caption{$\Delta t = 1.5$}
    \label{fig:q4-2}
  \end{subfigure}
  
  \caption{解の挙動}
  \label{fig:q6_re}
\end{figure}


\section{課題5 常微分方程式の爆発解}
\label{sec:q5}
\ref{sec:q4}章と同じく, $t \in [0, \infty]$における常微分方程式の初期値問題
\begin{gather}
  u' = -2u + 1, \ u(0) = 1
\end{gather}
を考える. \\
以下の陽解法
\begin{align}
  u_n &= u_{n-2} + 2 f(t_{n-1}, u_{n-1})\Delta t \\
  &= -4 \Delta t u_{n-1} + u_{n-2} + 2\Delta t
\end{align}
を用いて数値解を求める. このとき特性方程式
\begin{gather}
  \lambda^2 + 4\Delta t \lambda - 1 = 0
\end{gather}
の解は
\begin{align}
  \lambda_1 &= -2\Delta t + \sqrt{4 \Delta t^2 + 1} \\
  \lambda_2 &= -2\Delta t - \sqrt{4 \Delta t^2 + 1}
\end{align}
である. ここで$\Delta t > 0$より
\begin{align}
  |\lambda_2| &= 2\Delta t + \sqrt{4 \Delta t^2 + 1} \\
  &> 0 + \sqrt{0 + 1} \\
  &= 1
\end{align}
であるので, この解法は常に不安定である. \\
実際の数値計算は$\Delta t = 0.1$として$t \in [0, 10]$の範囲で実行した. 
その結果は図\ref{fig:q5}の通り. 
真の解は収束しているのに対し, この計算方法で求めた数値解は指数関数のオーダーで発散している. 

\begin{figure}[ht]
  \centering
  \includegraphics[width=0.8\textwidth]{question5/q5.png}
  \caption{爆発解の挙動}
  \label{fig:q5}
\end{figure}


\section{課題6 常微分方程式の初期値による解の振舞いの差異}
\label{sec:q6}
$t \in [0, \infty]$における常微分方程式の初期値問題
\begin{gather}
  u' = (u - 1) u, \ u(0) = u_0
\end{gather}
を考える. 
数値計算は$\Delta t = 0.01$として$t \in [0, 1.2]$の範囲で, 
初期値を$u_0 = 0.5, 1.0, 1.5$の3通りで実行した. その結果は図\ref{fig:q6}の通り.
$u_0 = 0.5, 1.0$の場合は一定の値に収束しているのに対し, $u_0 = 1.5$の場合は$t = 1.1$付近で急激に値が増加している. 

\begin{figure}[ht]
  \centering
  \includegraphics[width=0.8\textwidth]{question6/q6.png}
  \caption{初期値による解の振舞いの差異}
  \label{fig:q6}
\end{figure}


\section{課題7 Lorenz方程式の解の振舞い}
\label{sec:q7}
連立常微分方程式である, Lorenz方程式
\begin{align}
  x' &= 10(y - x), \ x(0) = 1 + \epsilon \\
  y' &= 28x - y - xz, \ y(0) = 0 \\
  z' &= xy - \frac{8}{3}z, \ z(0) = 0
\end{align}
を考える. 

\subsection{解の非周期的な長時間挙動}
\label{sec:7-1}
本節では$\epsilon = 0$として考える. 
時間に対する$x(t)$挙動及び$(x(t), y(t), z(t))$の挙動は, それぞれ図\ref{fig:q7-1a}, \ref{fig:q7-1b}の通り. 
確かに解の挙動は複雑で, カオスになっていることが確認できる. 

\begin{figure}[ht]
  \centering
  \includegraphics[width=0.8\textwidth]{question7/q7-1a.png}
  \caption{時間に対する$x(t)$の挙動}
  \label{fig:q7-1a}
\end{figure}

\begin{figure}[ht]
  \centering
  \includegraphics[width=0.8\textwidth]{question7/q7-1b.png}
  \caption{$(x(t), y(t), z(t))$の挙動}
  \label{fig:q7-1b}
\end{figure}

\newpage
\subsection{数値計算の誤差が解に与える影響}
\label{sec:7-2}
本節では$\epsilon = 0$として考える. 
$\Delta t = 0.01 \times 2^{-i}, \ i = 0, 1, ... , 13$として, 
前進Euler法とRunge-Kutta法を用いてLorenz方程式の数値解を計算した. 
$t = 15, 30, 60$における$i$と数値解$x(t)$の関係は, 図\ref{fig:q7-2a}, \ref{fig:q7-2b}, \ref{fig:q7-2c}の通り. 
$t = 15$では$i$が大きくなる, つまり$\Delta t$が小さくなるにつれて, $x(t)$は一定の値に収束している. 
$t = 30$ではRunge-Kutta法で計算した数値解は同様に収束しているが, 前進Euler法で計算した数値解は値が定まっていない. 
$t = 60$ではいずれの方法で求めた数値解も値が定まっていない. 
このことから, 数値計算の誤差が解に与える影響は$t$が大きくなるほど大きくなっているといえる. 
また誤差の影響の受けにくさはRunge-Kutta法の方が優れているといえる. 

\begin{figure}[ht]
  \centering
  \includegraphics[width=0.8\textwidth]{question7/q7-2a.png}
  \caption{$t = 15$における$i$と数値解$x(t)$の関係}
  \label{fig:q7-2a}
\end{figure}
\begin{figure}[ht]
  \centering
  \includegraphics[width=0.8\textwidth]{question7/q7-2b.png}
  \caption{$t = 30$における$i$と数値解$x(t)$の関係}
  \label{fig:q7-2b}
\end{figure}
\begin{figure}[ht]
  \centering
  \includegraphics[width=0.8\textwidth]{question7/q7-2c.png}
  \caption{$t = 60$における$i$と数値解$x(t)$の関係}
  \label{fig:q7-2c}
\end{figure}


\clearpage
\subsection{初期値の微小な変化が解に与える影響}
\label{sec:7-3}
Lorenz方程式の初期値のパラメータである$\epsilon$の値を$\epsilon = 0, 0.1, 0.01, 0.001$と変化させて解を計算した. 
数値計算は$\Delta t = 0.001$として$t \in [0, 100]$の範囲で実行した. 
数値計算の結果は図\ref{fig:q7-3}の通り. 
初期値はほとんど変化していないが数値解の振舞いは大きく異なっており, カオスが確認できる. 

\begin{figure}[ht]
  \centering
  \includegraphics[width=0.8\textwidth]{question7/q7-3.png}
  \caption{初期値を微小に変化させたときの解の振舞い}
  \label{fig:q7-3}
\end{figure}

\clearpage
\section{実行環境}
\label{sec:env}
本課題の実行環境は表\ref{tab:env}の通り.

\begin{table}[ht]
  \centering
  \caption{実行環境}
  \label{tab:env}
  \begin{tabular}[ht]{c c}
    \hline
    OS & Ubuntu 22.04.5 LTS \\
    CPU & 13th Gen Intel(R) Core(TM) i7-1360P \\
    メモリ & 8GB \\
    使用言語 & julia 1.12.1 \\
    \hline
  \end{tabular}
\end{table}

\section{参考文献}
\label{sec:ref}
数値計算のためのコードで誤っている箇所の修正, 及びより良い書き方の勉強のため, 
Gemini 2.5 Pro及びGemini 3.0 Proを使用した.


\end{document}