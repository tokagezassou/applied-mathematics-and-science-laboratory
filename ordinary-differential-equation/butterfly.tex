\documentclass[uplatex,a4j]{jsarticle}
\usepackage[dvipdfmx]{graphicx}
\usepackage{amsmath,ascmac}
\usepackage{bm}
\usepackage{algorithm,algorithmic}
\usepackage{listings}
\usepackage{subcaption}
\usepackage{amsfonts}
\usepackage{amssymb}
\usepackage{amsmath}
\lstset{%
  language={C},
  basicstyle={\small},%
  identifierstyle={\small},%
  commentstyle={\small\itshape},%
  keywordstyle={\small\bfseries},%
  ndkeywordstyle={\small},%
  stringstyle={\small\ttfamily},
  frame={tb},
  breaklines=true,
  columns=[l]{fullflexible},%
  numbers=left,%
  xrightmargin=0zw,%
  xleftmargin=3zw,%
  numberstyle={\scriptsize},%
  stepnumber=1,
  numbersep=1zw,%
  lineskip=-0.5ex%
}
\renewcommand{\lstlistingname}{コード}
\renewcommand{\lstlistlistingname}{コード目次}
\renewcommand{\thesubsection}{(\arabic{subsection})}
\renewcommand{\thesection}{\arabic{section}}

\begin{document}

\begin{titlepage}
    \centering
    \vfill
    
    {\Huge 数理工学実験\par}
    \vspace{1cm}
    
    {\Large 常微分方程式レポート\par}
    
    \vfill
    
    {\Large
    所属: 工学部情報学科数理工学コース2年\par
    学籍番号: 1029-36-1263\par
    氏名: 天野 塁\par
    }
    
    \vfill
    
    {\Large 提出日: \today \par}
    
    \vfill
\end{titlepage}

\section{課題1 身の回りの常微分方程式}
\label{sec:q1}

上空から落下する雨粒の運動を考える. 雨粒の位置を$x$, 速度を$v$, 質量を$m$とし, 重力加速度を$g$とする. 
また落下する雨粒に働く空気抵抗は速度に比例し, $-kv \ (k > 0)$と表されるとする. 
また初期位置及び初期速度を$x_0, v_0$とする.
このとき雨粒の運動は 
\begin{align}
  \begin{cases}
    x' = v \\
    m v' = mg - kv \\
    x(0) = x_0 \\
    v(0) = v_0
  \end{cases}
\end{align}
で表される. 


\section{課題2 常微分方程式の数値計算法の構成}
\label{sec:q2}
4次のAdams-Bashforth法及び4次のAdams-Moulton法は式\ref{eq:ab}, \ref{eq:am}の通り. 
ただし特に引数を明記しない限り$f_k = f_k(t_k, u_k)$とする. 
\begin{align}
  u_n &\approx u_{n-1} + \frac{\Delta t}{24} (55f_{n-1} - 59f_{n-2} + 37f_{n-3} - 9f_{n-4})
  \label{eq:ab} \\
  u_n &\approx u_{n-1} + \frac{\Delta t}{24} (9f_n + 19f_{n-1} - 5f_{n-2} + f_{n-3})
  \label{eq:am}
\end{align}
また式\ref{eq:ab}, \ref{eq:am}を用いた予測子・修正子法は式\ref{eq:q2}の通り. 

\begin{align}
  \begin{cases}
    u_n^* &\approx u_{n-1} + \frac{\Delta t}{24} (55f_{n-1} - 59f_{n-2} + 37f_{n-3} - 9f_{n-4}) \\
    u_n &\approx u_{n-1} + \frac{\Delta t}{24} (9f(t_n, u_n^*) + 19f_{n-1} - 5f_{n-2} + f_{n-3})
  \end{cases}
  \label{eq:q2}
\end{align}


\newpage
\section{実行環境}
\label{sec:env}
本課題の実行環境は表\ref{tab:env}の通り.

\begin{table}[ht]
  \centering
  \caption{実行環境}
  \label{tab:env}
  \begin{tabular}[ht]{c c}
    \hline
    OS & Ubuntu 22.04.5 LTS \\
    CPU & 13th Gen Intel(R) Core(TM) i7-1360P \\
    メモリ & 8GB \\
    使用言語 & julia 1.12.1 \\
    \hline
  \end{tabular}
\end{table}

\section{参考文献}
\label{sec:ref}
数値計算のためのコードで誤っている箇所の修正, 及びより良い書き方の勉強のため, 
Gemini 2.5 Pro及びGemini 3.0 Proを使用した.


\end{document}