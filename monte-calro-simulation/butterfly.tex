\documentclass[uplatex,a4j]{jsarticle}
\usepackage[dvipdfmx]{graphicx}
\usepackage{amsmath,ascmac}
\usepackage{bm}
\usepackage{algorithm,algorithmic}
\usepackage{listings}
\usepackage{subcaption}
\usepackage{amsfonts}
\usepackage{amssymb}
\usepackage{amsmath}
\lstset{%
  language={C},
  basicstyle={\small},%
  identifierstyle={\small},%
  commentstyle={\small\itshape},%
  keywordstyle={\small\bfseries},%
  ndkeywordstyle={\small},%
  stringstyle={\small\ttfamily},
  frame={tb},
  breaklines=true,
  columns=[l]{fullflexible},%
  numbers=left,%
  xrightmargin=0zw,%
  xleftmargin=3zw,%
  numberstyle={\scriptsize},%
  stepnumber=1,
  numbersep=1zw,%
  lineskip=-0.5ex%
}
\renewcommand{\lstlistingname}{コード}
\renewcommand{\lstlistlistingname}{コード目次}
\renewcommand{\thesubsection}{(\arabic{subsection})}
\renewcommand{\thesection}{\arabic{section}}

\begin{document}

\begin{titlepage}
    \centering
    \vfill
    
    {\Huge 数理工学実験\par}
    \vspace{1cm}
    
    {\Large モンテカルロシミュレーションレポート\par}
    
    \vfill
    
    {\Large
    所属: 工学部情報学科数理工学コース2年\par
    学籍番号: 1029-36-1263\par
    氏名: 天野 塁\par
    }
    
    \vfill
    
    {\Large 提出日: \today \par}
    
    \vfill
\end{titlepage}

\section{問題1 疫病の確率的な感染モデル}
\label{sec:q1}

人が1列に並んだ状態で, 場所$i$にいる人の状態を, 状態変数$s_i$を用いて表す. 
疫病にかかった状態を$s_i = 1$, かかっていない状態を$s_i = 0$とする. 
また時刻$t+1$における感染状況は前の時刻$t$での場所$[i-1, i, i+1]$における感染状況
$[s_{i-1}, s_i, s_{i+1}]$にのみ依存して確率的に決まるとする. 
ただし境界条件は周期的であるとする. 
$n_i = s_{i-1} + s_{i+1}$としたとき, 次の時刻において$s_i(t+1) = 1$となる確率$P(1|s_i, n_i)$は, 
パラメータ$p$を用いて以下の式\eqref{eq:q1}の通りに定義する. 

\begin{align}
  \begin{cases}
    P(1|0, 0) &= 0 \ \text{吸収状態} \\
    P(1|0, 1) &= p^2 \\
    P(1|0, 2) &= p^2 (2 - p^2) \\
    P(1|1, 0) &= p^2 (2 - p^2) \\
    P(1|1, 1) &= p^2 (p^3 - 2p^2 - p + 3) \\
    P(1|1, 2) &= p^2 (2 - p) (p^3 - 2p^2 + 2) \\
  \end{cases}
  \label{eq:q1}
\end{align}

\subsection{サンプルごとの感染者数の推移}
\label{sec:q1-1}

システムサイズ$64$, 最終時刻$256$, $p = 0.70$とし, 初期状態は1人だけが感染しているという条件の下で, 
サンプル状態列を10個作成した. 
各時刻における感染者数を表したグラフは図\ref{fig:q1-1}の通り. 
どのサンプルでも感染者数の推移は似ており, 時刻$80$あたりまでは増加し続け, 
それ以降は概ね$40$から$60$の間を振動している. 

\begin{figure}[ht]
  \centering
  \includegraphics[width=0.8\textwidth]{question1/q1-1.png}
  \caption{サンプルにごとの感染者数}
  \label{fig:q1-1}
\end{figure}

\subsection{パラメータ$p$による感染者数の平均値の推移}
\label{sec:q1-2}

システムサイズ$64$, 最終時刻$256$, 初期状態は1人だけが感染しているという条件の下で, 
$p = 0.64, 0.66, 0.68, 0.70$のそれぞれの場合についてサンプル状態列を10000個作成した. 
各時刻における感染者数のサンプル数$M = 100000$のサンプル平均を表したグラフは図\ref{fig:q1-2}の通り. 
感染者数のサンプル平均はどのパラメータ$p$の値の場合でも収束が確認できた. 
パラメータ$p$の値が大きいほど収束先の値は大きくなっており, また収束速度も速くなっている. 

\begin{figure}[ht]
  \centering
  \includegraphics[width=0.8\textwidth]{question1/q1-2.png}
  \caption{サンプルにごとの感染者数}
  \label{fig:q1-2}
\end{figure}


\section{問題2 円周率の計算}
\label{sec:q2}

\subsection{サンプル数による推定値と精度の比較}
\label{sec:q2-1}

以下の式\eqref{eq:q2-int}の定積分にモンテカルロ法を適用し, 円周率を計算することを考える. 
\begin{gather}
  \int_{0}^{1} f(x) dx = \int_{0}^{1} \frac{4}{1 + x^2} dx = \pi
  \label{eq:q2-int}
\end{gather}
区間$[0,1]$の一様分布に従うサンプル列${X_t}$を作成し, 円周率を
\begin{gather}
  \pi = \langle \frac{4}{1 + X^2} \rangle
\end{gather}
のように期待値とみなすことで, モンテカルロ法を用いて円周率を計算した. \\
計算で求めた推定値及び精度は, 独立なモンテカルロ計算の計算回数$N_M$と平均値の集合${\bar{f}_i}$を用いて
\begin{align}
  &推定値 \ \bar{\bar{f}} = \frac{\sum_{i=1}^{N_M} \bar{f}_i}{N_M}
  \label{eq:q2-est} \\
  &精度 \ \bar{s} = \sqrt{\frac{\sum_{i} (\bar{f}_i - \bar{\bar{f}})^2}{N_M (N_M - 1)}}
  \label{eq:q2-acc}
\end{align}
で与えられる. 
サンプル数を$10, 10^3, 10^5, 10^7$とした独立な計算を10回行うことで求めた, 
式\eqref{eq:q2-est}, \eqref{eq:q2-acc}の推定値及び精度は以下の表\ref{tab:q2-1}の通り. 
確かにサンプル数$N_M$が大きくなると, 精度の値は$\frac{1}{\sqrt{N_M}}$のオーダーで減少している. 

\begin{table}[ht]
  \centering
  \caption{サンプル数ごとの推定値と精度}
  \label{tab:q2-1}
  \begin{tabular}[ht]{c c c}
    \hline
    サンプル数 & 推定値 & 精度 \\
    \hline
    $10$ & 3.14448 & $9.32622 \times 10^{-2}$ \\
    $10^3$ & 3.14760 & $8.52686 \times 10^{-3}$ \\
    $10^3$ & 3.14097 & $5.99963 \times 10^{-4}$ \\
    $10^3$ & 3.14175 & $6.52591 \times 10^{-5}$ \\
    \hline
  \end{tabular}
\end{table}

\subsection{モンテカルロ法の計算精度の妥当性}
\label{sec:q2-2}

期待値を用いたモンテカルロ法の計算精度$s$は, サンプル数$M$を用いて解析的に
\begin{align}
  s &= \frac{1}{M} [ \langle f^2(X) \rangle  - {\langle f(X) \rangle}^2] \\
  &= \frac{1}{M} [\int_{0}^{1} (\frac{4}{1 + x^2})^2 dx - (\int_{0}^{1} \frac{4}{1 + x^2} dx)^2] \\
  &= \frac{2\pi + 4 - \pi^2}{M}
\end{align}
と求められる. 
これの真の精度と式\eqref{eq:q2-acc}の$\bar{s}$との比$\frac{\bar{s}}{s}$は表\ref{tab:q2-2}の通り. 
サンプル数が多くなるほど精度の比は$1$に近づいているため, 
サンプル数を十分多くとれば, 式\eqref{eq:q2-acc}で計算される精度の値は妥当だといえる. 

\begin{table}[ht]
  \centering
  \caption{サンプル数ごとの精度の比}
  \label{tab:q2-2}
  \begin{tabular}[ht]{c c c}
    \hline
    サンプル数 & 精度の比 \\
    \hline
    $10$ & 1.45019 \\
    $10^3$ & 1.32589 \\
    $10^3$ & 0.932920 \\
    $10^3$ & 1.01475 \\
    \hline
  \end{tabular}
\end{table}


\section{問題3 マルコフ過程の釣り合い条件}
\label{sec:q3}
次の時刻における状態が現在の状態のみに依存するマルコフ過程において, 
遷移確率$P(X \to Y)$とマルコフ過程の確率分布$W$は, 以下の式\eqref{eq:q3-balance}の釣り合い条件
\begin{gather}
  ^{\forall} X, \ W(X) = \sum_{Y} P(Y \to X) W(Y)
  \label{eq:q3-balance}
\end{gather}
を満たす. さらに一部の特殊な場合では,  以下の式\eqref{eq:q3-detailed-balance}の詳細釣り合い条件
\begin{gather}
  ^{\forall} X, ^{\forall} Y, \ P(X \to Y) W(X) = P(Y \to X) W(Y)
  \label{eq:q3-detailed-balance}
\end{gather}
を満たす. 

\subsection{メトロポリス法の詳細釣り合い条件}
\label{sec:q3-1}

メトロポリス法の遷移確率が式\eqref{eq:q3-detailed-balance}の詳細釣り合い条件を満たすことを示す. \\
メトロポリス法の遷移確率は, 確率分布$Q(X(t), X(t+1))$を用いて以下の式\eqref{eq:q3-1}で与えられる. 
\begin{align}
  P(X \to Y) &= \min{[1, \frac{W(Y)}{W(X)}]} Q(X, Y) \\
  P(Y \to X) &= \min{[1, \frac{W(X)}{W(Y)}]} Q(Y, X) \\
  \label{eq:q3-1}
\end{align}

$W(X) \geq W(Y)$のとき, 遷移確率は
\begin{align}
  P(X \to Y) &= \frac{W(Y)}{W(X)} Q(X, Y) \\
  P(Y \to X) &= Q(Y, X)
\end{align}
であり, $Q(X, Y) = Q(Y, X)$なので
\begin{align}
  P(X \to Y) &= \frac{W(Y)}{W(X)} Q(X, Y) \\
  &= \frac{W(Y)}{W(X)} P(Y \to X) \\
  \therefore P(X \to Y) W(X) &= P(Y \to X) W(Y)
\end{align}

$W(X) < W(Y)$のとき, 遷移確率は
\begin{align}
  P(X \to Y) &= Q(X, Y) \\
  P(Y \to X) &= \frac{W(X)}{W(Y)} Q(Y, X)
\end{align}
であり, $Q(X, Y) = Q(Y, X)$なので
\begin{align}
  P(Y \to X) &= \frac{W(X)}{W(Y)} Q(Y, X) \\
  &= \frac{W(X)}{W(Y)} P(Y \to X) \\
  \therefore P(X \to Y) W(X) &= P(Y \to X) W(Y)
\end{align}
以上よりメトロポリス法の遷移確率は式\eqref{eq:q3-detailed-balance}の詳細釣り合い条件を満たす. 


\subsection{熱浴法の詳細釣り合い条件}
\label{sec:q3-2}

熱浴法の遷移確率が式\eqref{eq:q3-detailed-balance}の詳細釣り合い条件を満たすことを示す. \\
状態$X(t) = (x_1, ... , x_i, ...)$から状態$X(t) = (x_1, ... , x'_i, ...)$へ遷移する
熱浴法の状態遷移確率は, 以下の式\eqref{eq:q3-2}で与えられる. 
\begin{gather}
  P(x_i \to x'_i) = \frac{W(x_1, ... , x'_i, ...)}{\sum_{x_i} W(x_1, ... , x_i, ...)}
  \label{eq:q3-2}
\end{gather}
で与えられる. \\
このとき状態$X(t) = (x_1, ... , x_i, ...)$から状態$X(t) = (x_1, ... , x''_i, ...)$を経て
状態$X(t) = (x_1, ... , x'_i, ...)$へ遷移する状態遷移確率は
\begin{align}
  P(x_i \to x''_i) P(x''_i \to x'_i)
  &= \sum_{x''_i} \frac{W(x_1, ... , x''_i, ...)} {\sum_{x_i} W(x_1, ... , x_i, ...)}
  \frac{W(x_1, ... , x'_i, ...)}{\sum_{x_i} W(x_1, ... , x''_i, ...)} \\
  &= \frac{\sum_{x''_i} W(x_1, ... , x''_i, ...)} {\sum_{x_i} W(x_1, ... , x_i, ...)}
  \frac{W(x_1, ... , x'_i, ...)}{\sum_{x_i} W(x_1, ... , x''_i, ...)} \\
  &= \frac{W(x_1, ... , x'_i, ...)}{\sum_{x_i} W(x_1, ... , x_i, ...)} \\
  &= P(x_i \to x'_i)
\end{align}
したがって状態$X(t) = (x_1, ... , x_i, ...)$が最終的に状態$X(t) = (x_1, ... , x'_i, ...)$へ遷移する確率は, 
途中経過によらず最初と最後の状態によってのみ定まる. \\
これを用いると2状態$X = (x_1, ... , x_i, ...)$, $Y = (y_1, ... , y_i, ...)$間の遷移確率は
\begin{align}
  P(X \to Y) &= \prod_{i} \frac{W(y_1, ... , y_i, ...)}{\sum_{x_i} W(x_1, ... , x_i, ...)} \\
  P(Y \to X) &= \prod_{i} \frac{W(x_1, ... , x_i, ...)}{\sum_{y_i} W(y_1, ... , y_i, ...)}
\end{align}
で与えられる. 2式の比をとって
\begin{align}
  \frac{P(X \to Y)}{P(Y \to X)}
  &= \prod_{i} \frac{W(y_1, ... , y_i, ...) \sum_{y_i} W(y_1, ... , y_i, ...)}
    {W(x_1, ... , x_i, ...) \sum_{x_i} W(x_1, ... , x_i, ...)} \\
  &= \prod_{i} \frac{W(y_1, ... , y_i, ...)}{W(x_1, ... , x_i, ...)}
\end{align}
$x_1, x_2, ...$と順番に状態が遷移するとしても一般性を失わないので
\begin{align}
  \frac{P(X \to Y)}{P(Y \to X)}
  &= \prod_{i} \frac{W(y_1, ... , y_i, ...)}{W(x_1, ... , x_i, ...)} \\
  &= \frac{W(y_1, x_2, x_3, ...)}{W(x_1, x_2, x_3, ...)}
    \frac{W(y_1, y_2, x_3, ...)}{W(y_1, x_2, x_3, ...)}
    \frac{W(y_1, y_2, y_3, ...)}{W(y_1, y_2, x_3, ...)} ... \\
  &= \frac{W(Y)}{W(X)} \\
  \therefore P(X \to Y) W(X) &= P(Y \to X) W(Y)
\end{align}
以上より熱浴法の遷移確率は式\eqref{eq:q3-detailed-balance}の詳細釣り合い条件を満たす. 

\subsection{詳細釣り合い条件と釣り合い条件の関係}
\label{sec:q3-3}

一般に遷移確率が式\eqref{eq:q3-detailed-balance}の詳細釣り合い条件を満たす場合は, 
式\eqref{eq:q3-balance}の釣り合い条件を満たすことを示す. \\
式\eqref{eq:q3-detailed-balance}の詳細釣り合い条件は
\begin{gather}
  ^{\forall} X, ^{\forall} Y, \ P(X \to Y) W(X) = P(Y \to X) W(Y)
\end{gather}
$Y$について左辺の和をとると
\begin{align}
  \sum_{Y} P(X \to Y) W(X) &= W(X) \sum_{Y} P(X \to Y) \\
  &= W(X)
\end{align}
したがって$^{\forall}X$に対して
\begin{gather}
  W(X) = \sum_{Y} P(Y \to X) W(Y)
\end{gather}
となるので, 式\eqref{eq:q3-balance}の釣り合い条件を満たす. 


\section{問題4 イジングモデルのモンテカルロシミュレーション}
\label{sec:q4}

\subsection{イジングモデルの熱浴法の手順}
\label{sec:q4-1}
イジングモデルの熱浴法の手順は以下の通り. \\
手順1: ランダムに格子点を1つ選び, 選んだ格子点の状態変数を$s$, フリップしたものを$s'$とする. \\
手順2: 以下の式\eqref{eq:q4-1}の確率$Q(s')$で$s$をフリップさせる. 
\begin{align}
  Q(s') &= \frac{P(..., s', ...)}{P(..., s, ...) + P(..., s', ...)} \\
  &= \frac{1}{\frac{P(..., s, ...)}{P(..., s', ...)} + 1}
  \label{eq:q4-1} \\
  \text{ただし} P(..., s, ...) &= \frac{e^{- \frac{E(S)}{T}}}{Z} \\
  E(S) &= -\sum_{i,j} s_i s_j \\
  Z &= \sum_{S} e^{- \frac{-E(S)}{T}}
\end{align}
なお実際の計算では効率を上げるため, 式を整理して
\begin{align}
  Q(s') &= \frac{1}{\frac{P(..., s, ...)}{P(..., s', ...)} + 1} \\
  &= \frac{1}{e^{\frac{E(S') - E(S)}{T}} + 1}
\end{align}
として, $Q(s')$をエネルギーの差$E(S') - E(S)$の関数として計算している. 

\subsection{イジングモデルの磁化の推移}
\label{sec:q4-2}
メトロポリス法及び熱浴法を用いて, 各時刻ごとの磁化
\begin{gather}
  m = \frac{1}{L^2} | \sum_{i}s_i |
\end{gather}
を計算した. その結果は図\ref{fig:q4-2}の通り. 
ただし温度は$T = \frac{2}{1 + z}, z = \sqrt{2}$, システムサイズは$L = 64$, 初期状態は全てのイジングスピンが1とした. 
いずれの方法でも磁化は一定の区間を振動しており, その振幅は熱浴法の方が小さい. 
そのため以降のイジングモデルの計算には熱浴法を使用した. 
\begin{figure}[ht]
  \centering
  \includegraphics[width=0.8\textwidth]{question4/q4-2.png}
  \caption{各時刻ごとの磁化の推移}
  \label{fig:q4-2}
\end{figure}

\subsection{磁化の期待値の温度とシステムサイズへの依存度}
\label{sec:q4-3}
イジングモデルの磁化の期待値の温度とシステムサイズへの依存度を求めるため, 
温度$T = \frac{2}{\log{(1 + z)}}$のパラメータ$z$を$z = 1.225, 1.250, ..., 1.500$の12通り, 
システムサイズ$L$を$L = 12, 16, 24, 32, 48, 64$の6通りのパターンで磁化$m$の期待値を計算した. 
図\ref{fig:q4-2}より, 多めに見積もっても時刻2000では初期値の影響が十分消えているといえる. 
そこで最初の2000ステップは磁化$m$の期待値に含めず, 以降の100000ステップの磁化$m$を期待値に含める, 
という計算を10回繰り返して, 全ての試行を含めた磁化$m$の期待値を計算した. 
その結果は図\ref{fig:q4-3}の通り. 
特に図\ref{fig:q4-3-1}が温度のパラメータ$z$と磁化の期待値との関係を, 
図\ref{fig:q4-3-2}がシステムサイズ$L$と磁化の期待値との関係を表している. 

\begin{figure}[htbp]
  \centering

  \begin{subfigure}{0.45\textwidth}
    \centering
    \includegraphics[width=\linewidth]{question4/q4-3-1.png}
    \caption{磁化の期待値の温度への依存度}
    \label{fig:q4-3-1}
  \end{subfigure}
  \hfill
  \begin{subfigure}{0.45\textwidth}
    \centering
    \includegraphics[width=\linewidth]{question4/q4-3-2.png}
    \caption{磁化の期待値のシステムサイズへの依存度}
    \label{fig:q4-3-2}
  \end{subfigure}
  
  \caption{磁化の期待値の温度とシステムサイズへの依存度}
  \label{fig:q4-3}
\end{figure}

\subsection{システムサイズが大きくなった極限での磁化の振る舞い}
\label{sec:q4-4}
図\ref{fig:q4-3-2}より, $z \leq 1.400$のときには磁化の期待値は減少しているが, 
$z \geq 1.425$とときには磁化の期待値は減少しなくなっている. 
温度の定義式が$T = \frac{2}{\log{(1 + z)}}$であり, $z$の短調減少関数であることに注意すると, 
システムサイズが大きくなった極限では, 温度$T$が高いときには磁化の期待値は0に収束すると考えられる.  
一方温度$T$がある一定の値より低くなると磁化の期待値は0.7から0.8付近の値に収束すると考えられ, 
ここに対称性の破れを見ることができる. \\
このように温度$T$の値によって磁化の期待値の振る舞いに相転移が見られ, 
臨界温度$\tilde{T}$は$\frac{2}{\log{(1 + 1.425)}} \leq \tilde{T} \leq \frac{2}{\log{(1 + 1.400)}}$
の範囲に存在するといえる. 
この臨界温度の近傍においては, 局所的に対称性が保たれた箇所と破られた箇所が混在しており, 
イジングスピンの向きが局所的に揃った塊がいたるところに見られると考えられる. 


\clearpage
\section{実行環境}
\label{sec:env}
本課題の実行環境は以下の表\ref{tab:env}の通り.

\begin{table}[ht]
  \centering
  \caption{実行環境}
  \label{tab:env}
  \begin{tabular}[ht]{c c}
    \hline
    OS & Ubuntu 22.04.5 LTS \\
    CPU & 13th Gen Intel(R) Core(TM) i7-1360P \\
    メモリ & 8GB \\
    使用言語 & julia 1.12.1 \\
    \hline
  \end{tabular}
\end{table}

\section{参考文献}
\label{sec:ref}
数値計算のためのコードで誤っている箇所の修正, 及びより良い書き方の勉強のため, 
Gemini 3.0 Proを使用した.


\end{document}