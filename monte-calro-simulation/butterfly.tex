\documentclass[uplatex,a4j]{jsarticle}
\usepackage[dvipdfmx]{graphicx}
\usepackage{amsmath,ascmac}
\usepackage{bm}
\usepackage{algorithm,algorithmic}
\usepackage{listings}
\usepackage{subcaption}
\usepackage{amsfonts}
\usepackage{amssymb}
\usepackage{amsmath}
\lstset{%
  language={C},
  basicstyle={\small},%
  identifierstyle={\small},%
  commentstyle={\small\itshape},%
  keywordstyle={\small\bfseries},%
  ndkeywordstyle={\small},%
  stringstyle={\small\ttfamily},
  frame={tb},
  breaklines=true,
  columns=[l]{fullflexible},%
  numbers=left,%
  xrightmargin=0zw,%
  xleftmargin=3zw,%
  numberstyle={\scriptsize},%
  stepnumber=1,
  numbersep=1zw,%
  lineskip=-0.5ex%
}
\renewcommand{\lstlistingname}{コード}
\renewcommand{\lstlistlistingname}{コード目次}
\renewcommand{\thesubsection}{(\arabic{subsection})}
\renewcommand{\thesection}{\arabic{section}}

\begin{document}

\begin{titlepage}
    \centering
    \vfill
    
    {\Huge 数理工学実験\par}
    \vspace{1cm}
    
    {\Large モンテカルロシミュレーションレポート\par}
    
    \vfill
    
    {\Large
    所属: 工学部情報学科数理工学コース2年\par
    学籍番号: 1029-36-1263\par
    氏名: 天野 塁\par
    }
    
    \vfill
    
    {\Large 提出日: \today \par}
    
    \vfill
\end{titlepage}

\section{問題1 疫病の確率的な感染モデル}
\label{sec:q1}

人が1列に並んだ状態で, 場所$i$にいる人の状態を, 状態変数$s_i$を用いて表す. 
疫病にかかった状態を$s_i = 1$, かかっていない状態を$s_i = 0$とする. 
また時刻$t+1$における感染状況は前の時刻$t$での場所$[i-1, i, i+1]$における感染状況
$[s_{i-1}, s_i, s_{i+1}]$にのみ依存して確率的に決まるとする. 
ただし境界条件は周期的であるとする. 
$n_i = s_{i-1} + s_{i+1}$としたとき, 次の時刻において$s_i(t+1) = 1$となる確率$P(1|s_i, n_i)$は, 
パラメータ$p$を用いて以下の式\ref{eq:q1}の通りに定義する. 

\begin{align}
  \begin{cases}
    P(1|0, 0) &= 0 \ \text{吸収状態} \\
    P(1|0, 1) &= p^2 \\
    P(1|0, 2) &= p^2 (2 - p^2) \\
    P(1|1, 0) &= p^2 (2 - p^2) \\
    P(1|1, 1) &= p^2 (p^3 - 2p^2 - p + 3) \\
    P(1|1, 2) &= p^2 (2 - p) (p^3 - 2p^2 + 2) \\
  \end{cases}
  \label{eq:q1}
\end{align}

\subsection{サンプルごとの感染者数の推移}
\label{sec:q1-1}

システムサイズ$64$, 最終時刻$256$, $p = 0.70$とし, 初期状態は1人だけが感染しているという条件の下で, 
サンプル状態列を10個作成した. 
各時刻における感染者数を表したグラフは図\ref{fig:q1-1}の通り. 
どのサンプルでも感染者数の推移は似ており, 時刻$80$あたりまでは増加し続け, 
それ以降は概ね$40$から$60$の間を振動している. 

\begin{figure}[ht]
  \centering
  \includegraphics[width=0.8\textwidth]{question1/q1-1.png}
  \caption{サンプルにごとの感染者数}
  \label{fig:q1-1}
\end{figure}

\subsection{パラメータ$p$による感染者数の平均値の推移}
\label{sec:q1-2}

システムサイズ$64$, 最終時刻$256$, 初期状態は1人だけが感染しているという条件の下で, 
$p = 0.64, 0.66, 0.68, 0.70$のそれぞれの場合についてサンプル状態列を10000個作成した. 
各時刻における感染者数のサンプル数$M = 100000$のサンプル平均を表したグラフは図\ref{fig:q1-2}の通り. 
感染者数のサンプル平均はどのパラメータ$p$の値の場合でも収束が確認できた. 
パラメータ$p$の値が大きいほど収束先の値は大きくなっており, また収束速度も速くなっている. 

\begin{figure}[ht]
  \centering
  \includegraphics[width=0.8\textwidth]{question1/q1-2.png}
  \caption{サンプルにごとの感染者数}
  \label{fig:q1-2}
\end{figure}



\clearpage
\section{実行環境}
\label{sec:env}
本課題の実行環境は表\ref{tab:env}の通り.

\begin{table}[ht]
  \centering
  \caption{実行環境}
  \label{tab:env}
  \begin{tabular}[ht]{c c}
    \hline
    OS & Ubuntu 22.04.5 LTS \\
    CPU & 13th Gen Intel(R) Core(TM) i7-1360P \\
    メモリ & 8GB \\
    使用言語 & julia 1.12.1 \\
    \hline
  \end{tabular}
\end{table}

\section{参考文献}
\label{sec:ref}
数値計算のためのコードで誤っている箇所の修正, 及びより良い書き方の勉強のため, 
Gemini 3.0 Proを使用した.


\end{document}