\documentclass[uplatex,a4j]{jsarticle}
\usepackage[dvipdfmx]{graphicx}
\usepackage{amsmath,ascmac}
\usepackage{bm}
\usepackage{algorithm,algorithmic}
\usepackage{listings}
\usepackage{subcaption}
\usepackage{amsfonts}
\lstset{%
  language={C},
  basicstyle={\small},%
  identifierstyle={\small},%
  commentstyle={\small\itshape},%
  keywordstyle={\small\bfseries},%
  ndkeywordstyle={\small},%
  stringstyle={\small\ttfamily},
  frame={tb},
  breaklines=true,
  columns=[l]{fullflexible},%
  numbers=left,%
  xrightmargin=0zw,%
  xleftmargin=3zw,%
  numberstyle={\scriptsize},%
  stepnumber=1,
  numbersep=1zw,%
  lineskip=-0.5ex%
}
\renewcommand{\lstlistingname}{コード}
\renewcommand{\lstlistlistingname}{コード目次}
\renewcommand{\thesubsection}{(\arabic{subsection})}
\renewcommand{\thesection}{課題\arabic{section}}

% ***************ここ変える*****************
\title{数値線形代数レポート}
\author{1029 36 1263 天野塁}
\date{2025/10/27}
% ****************************************

\begin{document}

\maketitle

\section{消去法を用いた連立一次方程式の解法}
\label{sec:q1}

\subsection{データのプロット}
\label{sec:q1_1}

大きさが$n = 100,200,400,800$の正方行列$A$とベクトル$\bm{b}$の組を100個ずつ生成して, 
それぞれのペアに対して消去法を用いて連立一次方程式
\begin{align}
  A \bm{x} = \bm{b}
\end{align}
の数値解を計算した. \\
各$n$に対する数値解$\bm{\tilde{x}}$の残差ノルム$\| \bm{b} - A \bm{\tilde{x}} \|$, 
真の解$\bm{x}$との相対誤差$\frac{\| \bm{x} - \bm{\tilde{x}} \|}{\| \bm{x} \|}$, 計算時間は
それぞれ図\ref{fig:q1_rn}, 図\ref{fig:q1_re}, 図\ref{fig:q1_ct}の通り. 
それぞれのグラフの直線はデータの中央値を表している.

\begin{figure}[htbp]
  \centering

  \begin{minipage}[t]{0.48\textwidth}
    \centering
    \includegraphics[width=\linewidth]{question1/q1_residual_norm.png}
    \captionof{figure}{残差ノルム}
    \label{fig:q1_rn}
  \end{minipage}
  \hfill
  \begin{minipage}[t]{0.48\textwidth}
    \centering
    \includegraphics[width=\linewidth]{question1/q1_relative_error.png}
    \captionof{figure}{相対誤差}
    \label{fig:q1_re}
  \end{minipage}
  
\end{figure}

\begin{figure}[ht]
  \centering
  \includegraphics[width=0.48\textwidth]{question1/q1_calculation_time.png}
  \caption{計算時間}
  \label{fig:q1_ct}
\end{figure}

\newpage
\subsection{数値解の精度と次数$n$の関係}
\label{sec:q1_2}
図\ref{fig:q1_rn}, 図\ref{fig:q1_re}において各$n$における中央値の幅が概ね一定なことから, 
残差ノルム及び相対誤差は多項式のオーダーで増加していると読み取れる. 
消去法の計算量は$O(n^3)$であり, 計算の丸め誤差が$O(n^3)$で蓄積されていくので, このような結果になったと考えられる.

\subsection{計算時間と次数$n$の関係}
\label{sec:q1_3}
図\ref{fig:q1_ct}において各$n$における中央値の幅が概ね一定なことから, 
計算時間は多項式のオーダーで増加していると読み取れる. 
さらに図\ref{fig:q1_rn}, 図\ref{fig:q1_re}と比べてデータのばらつきが大幅に少なくなっている.
消去法の計算量は$O(n^3)$であり, 最初の行列及びベクトルにはよらないので, このような結果になったと考えられる.


\section{LU分解を用いた連立一次方程式の解法}
\label{sec:q2}

\subsection{データのプロット}
\label{sec:q2_1}

大きさが$n = 100,200,400,800$の正方行列$A$とベクトル$\bm{b}$の組を100個ずつ生成して, 
それぞれのペアに対してLU分解を用いて連立一次方程式
\begin{align}
  A \bm{x} = \bm{b}
\end{align}
の数値解を計算した. \\
各$n$に対する数値解$\bm{\tilde{x}}$の残差ノルム$\| \bm{b} - A \bm{\tilde{x}} \|$, 
真の解$\bm{x}$との相対誤差$\frac{\| \bm{x} - \bm{\tilde{x}} \|}{\| \bm{x} \|}$, 計算時間は
それぞれ図\ref{fig:q1_rn}, 図\ref{fig:q1_re}, 図\ref{fig:q1_ct}の通り. 
それぞれのグラフの直線はデータの中央値を表している.

\begin{figure}[htbp]
  \centering

  \begin{minipage}[t]{0.48\textwidth}
    \centering
    \includegraphics[width=\linewidth]{question2/q2_residual_norm.png}
    \captionof{figure}{残差ノルム}
    \label{fig:q2_rn}
  \end{minipage}
  \hfill
  \begin{minipage}[t]{0.48\textwidth}
    \centering
    \includegraphics[width=\linewidth]{question2/q2_relative_error.png}
    \captionof{figure}{相対誤差}
    \label{fig:q2_re}
  \end{minipage}
  
\end{figure}

\begin{figure}[ht]
  \centering
  \includegraphics[width=0.48\textwidth]{question2/q2_calculation_time.png}
  \caption{計算時間}
  \label{fig:q2_ct}
\end{figure}

\newpage
\subsection{数値解の精度と次数$n$の関係}
\label{sec:q2_2}
図\ref{fig:q2_rn}, 図\ref{fig:q2_re}において各$n$における中央値の幅が概ね一定なことから, 
残差ノルム及び相対誤差は多項式のオーダーで増加していると読み取れる. 
LU分解の計算量は$O(n^3)$であり, 計算の丸め誤差が$O(n^3)$で蓄積されていくので, このような結果になったと考えられる.

\subsection{計算時間と次数$n$の関係}
\label{sec:q2_3}
図\ref{fig:q2_ct}において各$n$における中央値の幅が概ね一定なことから, 
計算時間は多項式のオーダーで増加していると読み取れる. 
さらに図\ref{fig:q2_rn}, 図\ref{fig:q2_re}と比べてデータのばらつきが大幅に少なくなっている.
LU分解の計算量は$O(n^3)$であり, 最初の行列及びベクトルにはよらないので, このような結果になったと考えられる.


\section{消去法とLU分解の比較}
図\ref{fig:q1_rn}と\ref{fig:q2_rn}, \ref{fig:q1_re}と\ref{fig:q2_re}, 
\ref{fig:q1_ct}と\ref{fig:q2_ct}の中央値を比較するとどれも概ね等しいので, 
消去法とLU分解で数値解の精度と計算速度は変わらないと読み取れる. 
実際どちらの方法においても, ボトルネックになっている計算はピボットを用いた掃き出しであり, 
それは加減算と除算をそれぞれ$O(n^3)$回ずつ行うものである. 
そのため消去法とLU分解で差が見られないと考えられる. 

\section{べき乗法を用いた第1固有ベクトルの計算}
\label{sec:q4}

\subsection{データのプロット}
\label{sec:q4_1}
大きさが$n = 100,200,400,800$の正方行列$A$を100個ずつ生成して, 
それぞれの行列に対してべき乗法を用いて第1固有ベクトルを計算した. \\
各$n$に対して, 数値計算で求めた第1固有値$\tilde{\lambda_1}$と第1固有ベクトル$\bm{\tilde{v_1}}$に対する
固有方程式の残差ノルム$\| \tilde{\lambda_1}\bm{\tilde{v_1}} - A \bm{\tilde{v_1}} \|$, 
$\tilde{\lambda_1}$の相対誤差$\frac{|\lambda_1 - \tilde{\lambda_1}|}{|\lambda_1|}$, 
$\bm{\tilde{v_1}}$の相対誤差$\frac{\| \bm{v_1} - \bm{\tilde{v_1}} \|}{\| \bm{v_1} \|}$, 
計算時間, 収束までに要した反復回数は, 
それぞれ図\ref{fig:q4_rn}, 図\ref{fig:q4_val_re}, 図\ref{fig:q4_vec_re}, 図\ref{fig:q4_ct}, 図\ref{fig:q4_ic}, の通り. 
それぞれのグラフの直線はデータの中央値を表している.

\begin{figure}[ht]
  \centering
  \includegraphics[width=0.48\textwidth]{question4/q4_residual_norm.png}
  \caption{残差ノルム}
  \label{fig:q4_rn}
\end{figure}

\begin{figure}[htbp]
  \centering

  \begin{minipage}[t]{0.48\textwidth}
    \centering
    \includegraphics[width=\linewidth]{question4/q4_eigenvalue_relative_error.png}
    \captionof{figure}{第1固有値の相対誤差}
    \label{fig:q4_val_re}
  \end{minipage}
  \hfill
  \begin{minipage}[t]{0.48\textwidth}
    \centering
    \includegraphics[width=\linewidth]{question4/q4_eigenvector_relative_error.png}
    \captionof{figure}{第1固有ベクトルの相対誤差}
    \label{fig:q4_vec_re}
  \end{minipage}
  
\end{figure}

\begin{figure}[htbp]
  \centering

  \begin{minipage}[t]{0.48\textwidth}
    \centering
    \includegraphics[width=\linewidth]{question4/q4_calculation_time.png}
    \captionof{figure}{計算時間}
    \label{fig:q4_ct}
  \end{minipage}
  \hfill
  \begin{minipage}[t]{0.48\textwidth}
    \centering
    \includegraphics[width=\linewidth]{question4/q4_iteration_count.png}
    \captionof{figure}{反復回数}
    \label{fig:q4_ic}
  \end{minipage}
  
\end{figure}

\newpage
\subsection{$\tilde{\lambda_1}, \bm{\tilde{v_1}}$の精度と次数$n$の関係}
\label{sec:q4_2}
図\ref{fig:q4_val_re}, 図\ref{fig:q4_vec_re}において各中央値が概ね等しいことから, 
次数$n$に関わらず$\tilde{\lambda_1}$と$\bm{\tilde{v_1}}$は常に等しい精度で計算できていると読み取れる. 
一方図\ref{fig:q4_rn}において各中央値は異なっており, その幅が概ね一定なことから, 
残差ノルムは次数$n$に対して多項式のオーダーで増加していると読み取れる. \\
$\tilde{\lambda_1}, \bm{\tilde{v_1}}$の精度が次数$n$によらないのは, 
べき乗法の収束判定を同じ閾値で判断してるからだと考えられる. 
今回は閾値を$1.0 \times 10^{-12}$に設定しており, 現に相対誤差は$10^{-13}$から$10^{-12}$のオーダーである. 
一方残差ノルムの計算量は$O(n^2)$であり, 第1固有値と第1固有ベクトルの計算時点からの誤差が$O(n^2)$で蓄積されていく. 
そのため残差ノルムの値は相対誤差よりも大きい$10^{-11}$のオーダーになり, 
さらに多項式のオーダーの有意な差が発生したと考えられる.



\end{document}