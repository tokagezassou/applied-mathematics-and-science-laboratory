\documentclass[uplatex,a4j]{jsarticle}
\usepackage[dvipdfmx]{graphicx}
\usepackage{amsmath,ascmac}
\usepackage{bm}
\usepackage{algorithm,algorithmic}
\usepackage{listings}
\usepackage{subcaption}
\usepackage{amsfonts}
\lstset{%
  language={C},
  basicstyle={\small},%
  identifierstyle={\small},%
  commentstyle={\small\itshape},%
  keywordstyle={\small\bfseries},%
  ndkeywordstyle={\small},%
  stringstyle={\small\ttfamily},
  frame={tb},
  breaklines=true,
  columns=[l]{fullflexible},%
  numbers=left,%
  xrightmargin=0zw,%
  xleftmargin=3zw,%
  numberstyle={\scriptsize},%
  stepnumber=1,
  numbersep=1zw,%
  lineskip=-0.5ex%
}
\renewcommand{\lstlistingname}{コード}
\renewcommand{\lstlistlistingname}{コード目次}
\renewcommand{\thesubsection}{(\arabic{subsection})}
\renewcommand{\thesection}{課題\arabic{section}}

% ***************ここ変える*****************
\title{数値線形代数レポート}
\author{1029 36 1263 天野塁}
\date{2025/10/27}
% ****************************************

\begin{document}

\maketitle

\section{消去法を用いた連立一次方程式の解法}
\label{sec:q1}

\subsection{データのプロット}
\label{sec:q1_1}

大きさが$n = 100,200,400,800$の対角行列$A$とベクトル$\bm{b}$の組を100個ずつ生成して, 
それぞれのペアに対して消去法を用いて連立一次方程式
\begin{align}
  A \bm{x} = \bm{b}
\end{align}
の数値解を計算した. \\
各$n$に対する数値解$\bm{\tilde{x}}$の残差ノルム$\| \bm{b} - A \bm{\tilde{x}} \|$, 
真の解$\bm{x}$との相対誤差$\frac{\| \bm{x} - \bm{\tilde{x}} \|}{\bm{x}}$, 計算時間は
それぞれ図\ref{fig:q1_rn}, 図\ref{fig:q1_le}, 図\ref{fig:q1_ct}の通り. 
それぞれのグラフの直線はデータの中央値を表している.

\begin{figure}[htbp]
  \centering

  \begin{minipage}[t]{0.48\textwidth}
    \centering
    \includegraphics[width=\linewidth]{question1/q1_residual_norm.png}
    \captionof{figure}{残差ノルム}
    \label{fig:q1_rn}
  \end{minipage}
  \hfill
  \begin{minipage}[t]{0.48\textwidth}
    \centering
    \includegraphics[width=\linewidth]{question1/q1_relative_error.png}
    \captionof{figure}{相対誤差}
    \label{fig:q1_le}
  \end{minipage}
  
\end{figure}

\begin{figure}[ht]
  \centering
  \includegraphics[width=0.48\textwidth]{question1/q1_calculation_time.png}
  \caption{計算時間}
  \label{fig:q1_ct}
\end{figure}

\newpage
\subsection{数値解の精度と次数$n$の関係}
\label{sec:q1_2}
図\ref{fig:q1_rn}, 図\ref{fig:q1_le}において各$n$における中央値の幅が一定なことから, 
残差ノルム及び相対誤差は多項式のオーダーで増加していると読み取れる. 
消去法の計算量は$O(n^3)$であり, 計算の丸め誤差が$O(n^3)$で蓄積されていくので, このような結果になったと考えられる.

\subsection{計算時間と次数$n$の関係}
\label{sec:q1_3}
図\ref{fig:q1_ct}において各$n$における中央値の幅が一定なことから, 
計算時間は多項式のオーダーで増加していると読み取れる. 
さらに図\ref{fig:q1_rn}, 図\ref{fig:q1_le}と比べてデータのばらつきが大幅に少なくなっている.
消去法の計算量は$O(n^3)$であり, 最初の行列及びベクトルにはよらないので, このような結果になったと考えられる.
\end{document}