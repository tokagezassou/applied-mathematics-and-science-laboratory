% !TEX program = lualatex
\documentclass[...]{ltjsarticle}
\usepackage{graphicx}
\usepackage{amsmath,ascmac}
\usepackage{bm}
\usepackage{algorithm,algorithmic}
\usepackage{listings}
\usepackage{subcaption}
\usepackage{amsfonts}
\usepackage{amssymb}
\usepackage{amsmath}
\usepackage{enumitem}
\lstdefinelanguage{Julia}%
  {morekeywords={abstract,break,case,catch,const,continue,do,else,elseif,end,export,false,for,function,immutable,import,importall,if,in,macro,module,quote,return,struct,true,try,type,typealias,using,while},%
   sensitive=true,%
   morecomment=[l]{\#},%      % # を \# に修正
   morecomment=[s]{\#=}{=\#},% % # を \# に修正
   morestring=[b]",%
   morestring=[m]'%
  }
\lstset{%
  language={Julia}, % Juliaをメインで使うならここを {Julia} にしてもOKです
  basicstyle={\small\ttfamily}, % 読みやすさのため ttfamily を推奨
  identifierstyle={\small},%
  commentstyle={\small\itshape},%
  keywordstyle={\small\bfseries},%
  ndkeywordstyle={\small},%
  stringstyle={\small\ttfamily},
  frame={tb},
  breaklines=true,
  columns=fullflexible, % [l] を外すとより安定します
  keepspaces=true, % スペースを維持
  numbers=left,%
  xrightmargin=0pt, % 0zw から 0pt に変更
  xleftmargin=3em,  % 3zw から 3em に変更(1zw ≒ 1em です)
  numberstyle={\scriptsize},%
  stepnumber=1,
  numbersep=1em,    % 1zw から 1em に変更
  lineskip=-0.5ex%
}
\renewcommand{\lstlistingname}{コード}
\renewcommand{\lstlistlistingname}{コード目次}
\renewcommand{\thesubsection}{(\arabic{subsection})}
\renewcommand{\thesection}{\arabic{section}}

\newcounter{algonum}
\newenvironment{algoitembox}[1]{%
  \refstepcounter{algonum}% カウンターを1進める
  \begin{itembox}[l]{[\thealgonum] #1}% タイトルに番号を表示
}{%
  \end{itembox}%
}


\begin{document}

\begin{titlepage}
    \centering
    \vfill
    
    {\Huge 数理工学実験\par}
    \vspace{1cm}
    
    {\Large 連続最適化レポート\par}
    
    \vfill
    
    {\Large
    所属: 工学部情報学科数理工学コース2年\par
    学籍番号: 1029-36-1263\par
    氏名: 天野 塁\par
    }
    
    \vfill
    
    {\Large 提出日: \today \par}
    
    \vfill
\end{titlepage}

\section{課題15 二分法及びニュートン法による零点探索}
\label{sec:q15}
次の関数$f: \mathbb{R} \to \mathbb{R}$
\begin{gather}
  f(x) := x^3 + 2x^2 - 5x - 6
\end{gather}
の零点を数値的に求める. 

\subsection{グラフの描画}
\label{sec:q15-1}
関数$y = f(x)$を$-10 \leq x \leq 10, \ -10 \leq y \leq 10$の範囲で描画すると, 図\ref{fig:q15-1}のようになる. 
グラフの概形を見ると, 零点は$x = -3, -1, 2$付近に存在していると読み取れる. 

\begin{figure}[ht]
  \centering
  \includegraphics[width=0.8\textwidth]{question15/q15.png}
  \caption{関数$y = f(x)$の概形}
  \label{fig:q15-1}
\end{figure}


\subsection{二分法による零点の探索}
\label{sec:q15-2}
二分法を用いて関数$f(x)$の零点を数値的に求める. 二分法のアルゴリズムは以下の通り. 

\begin{algoitembox}{二分法}
  \label{algo:bisec}
  \begin{description}
    \item[step 0:]$f(a_0)<0, f(b_0) \geq 0$を満たす初期点$a_0, b_0$を選び, 
    終了条件$\epsilon > 0$を決める. 
    \item[step 1:] 中間点 $c_n := (a_n + b_n)/2$ を求める. 
    求めた$c_n$が終了条件$|f(c_n)| \leq \epsilon$を満たせば, $c_n$を零点として終了する. 
    \item[step 2:]$f(c_n) < 0$であれば$a_{n+1} = c_n,\ b_{b+1} = b_n$とし, 
    $f(c_n) \geq 0$であれば$a_{n+1} = a_n,\ b_{b+1} = c_n$として, step 1へ戻る. 
  \end{description}
\end{algoitembox}

図\ref{fig:q15-1}のグラフの概形を元に, 初期値の組$a_0, b_0$の間にちょうど1つだけ零点が含まれるように初期値を設定した. 
二分法で求めた関数$f(x)$の零点とその初期値の組は以下の表\ref{tab:q15-2}の通り. 

\begin{table}[ht]
  \centering
  \caption{二分法で求めた零点}
  \label{tab:q15-2}
  \begin{tabular}[ht]{c c}
    \hline
    零点 & 初期値の組 \\
    \hline
    $-3.0000$ & $(a_0, b_0) = (-3.3, -2.7)$ \\
    $-1.0000$ & $(a_0, b_0) = (-0.7, -1.3)$ \\
    $2.0000$ & $(a_0, b_0) = (1.7, 2.3)$ \\
    \hline
  \end{tabular}
\end{table}


\subsection{ニュートン法による零点の探索}
\label{sec:q15-3}
ニュートン法を用いて関数$f(x)$の零点を数値的に求める. ニュートン法のアルゴリズムは以下の通り. 

\begin{algoitembox}{ニュートン法}
  \label{algo:old_newton}
  \begin{description}
    \item[step 0:]初期点$x_0$を選び, 終了条件$\epsilon > 0$を決める. 
    \item[step 1:]$\Delta x_n = \frac{-f(x_n)}{f'(x_n)}$とし, 
    $x_{n+1} = x_n + \Delta x_n$とする. 
    \item[step 2:]求めた$x_{n+1}$が終了条件$|f(x_{n+1})| \leq \epsilon$を満たせば, $x_{n+1}$を零点として終了する. 
    満たさなければstep 1に戻る. 
  \end{description}
\end{algoitembox}

図\ref{fig:q15-1}のグラフの概形を元に, 零点の近くに初期値を設定した. 
ニュートン法で求めた関数$f(x)$の零点とその初期値の組は以下の表\ref{tab:q15-3}の通り. 

\begin{table}[ht]
  \centering
  \caption{ニュートン法で求めた零点}
  \label{tab:q15-3}
  \begin{tabular}[ht]{c c}
    \hline
    零点 & 初期値 \\
    \hline
    $-3.0000$ & $x_0 = -3.3$ \\
    $-1.0000$ & $x_0 = -1.3$ \\
    $2.0000$ & $x_0 = 2.3$ \\
    \hline
  \end{tabular}
\end{table}


\clearpage
\section{課題16 最急降下法及びニュートン法による停留点探索}
\label{sec:q16}
次の関数$f: \mathbb{R} \to \mathbb{R}$
\begin{gather}
  f(x) := \frac{1}{3} x^3 - x^2 - 3x + \frac{5}{3}
\end{gather}
の零点を数値的に求める. 


\subsection{最急降下法による停留点の探索}
\label{sec:q16-1}
最急降下法を用いて関数$f(x)$の停留点を数値的に求める. 最急降下法のアルゴリズムは以下の通り. 
ただしステップサイズは一括で$t_n = \frac{1}{n+1}$とする. 

\begin{algoitembox}{最急降下法}
  \label{algo:sdm}
  \begin{description}
    \item[step 0:]初期点$x_0$を選び, 終了条件$\epsilon > 0$を決める. 
    \item[step 1:]$d_n = - \nabla f(x_n)$とし, ステップサイズ$t_n$を決定する. 
    \item[step 2:]$x_{n+1} = x_n + t_n d_n$とし, 
    求めた$x_{n+1}$が終了条件$|f(x_{n+1})| \leq \epsilon$を満たせば, $x_{n+1}$を零点として終了する. 
    満たさなければstep 1に戻る. 
  \end{description}
\end{algoitembox}

最急降下法で求めた関数$f(x)$の停留点とその初期値の組は以下の表\ref{tab:q16-1}の通り. 

\begin{table}[ht]
  \centering
  \caption{最急降下法で求めた停留点}
  \label{tab:q16-1}
  \begin{tabular}[ht]{c c}
    \hline
    停留点 & 初期値 \\
    $3.0000$ & $x_0 = 0.50000$ \\
    \hline
  \end{tabular}
\end{table}


\subsection{ニュートン法による停留点の探索}
\label{sec:q16-2}
最急降下法を用いて関数$f(x)$の停留点を数値的に求める. ニュートン法のアルゴリズムは以下の通り. 
ただしステップサイズは一括で$t_n = 1.0$とする. 

\begin{algoitembox}{ニュートン法}
  \label{algo:newton}
  \begin{description}
    \item[step 0:]初期点$x_0$を選び, 終了条件$\epsilon > 0$を決める. 
    \item[step 1:]$d_n = - \nabla^2 f(x_n)^{-1} \nabla f(x_n)$とし, ステップサイズ$t_n$を決定する. 
    \item[step 2:]$x_{n+1} = x_n + t_n d_n$とし, 
    求めた$x_{n+1}$が終了条件$|f(x_{n+1})| \leq \epsilon$を満たせば, $x_{n+1}$を零点として終了する. 
    満たさなければstep 1に戻る. 
  \end{description}
\end{algoitembox}

\clearpage
ニュートン法で求めた関数$f(x)$の停留点とその初期値の組は以下の表\ref{tab:q16-2}の通り. 

\begin{table}[ht]
  \centering
  \caption{ニュートン法で求めた停留点}
  \label{tab:q16-2}
  \begin{tabular}[ht]{c c}
    \hline
    停留点 & 初期値 \\
    $3.0000$ & $x_0 = 5.0000$ \\
    \hline
  \end{tabular}
\end{table}


\section{課題17 2次元の最適化問題の準備}
\label{sec:q17}
次の2次元の最適化問題
\begin{alignat}{2}
  \label{eq:q17}
  &\text{minimize} \quad && f(x) = x_0^2 + e^{x_0} + x_1^4 + x_1^2 - 2 x_0 x_1 + 3 \\
  &\text{subject to} \quad  &&x = (x_0, x_1)^\top \in \mathbb{R}^2
\end{alignat}
の解を数値的に求める. そのためにまずは必要な関数を実装する. 


\subsection{xを入力として$f(x)$を出力する関数の作成}
\label{sec:q17-1}
点$x$を引数にとり, 目的関数の値$f(x)$を返すjuliaの関数は以下のコード\ref{code:1}の通り.

\lstinputlisting[
  caption={f(x)を出力する関数},
  label={code:1},
  linerange={1-5}
]{question18/question18.jl}


\subsection{xを入力として$\nabla f(x)$を出力する関数の作成}
\label{sec:q17-2}
勾配ベクトル$\nabla f(x)$は
\begin{gather}
  \nabla f(x) =
  \begin{pmatrix}
    2 x_0 + e^{x_0} - 2 x_1 \\
    4 x_1^3 + 2 x_1 - 2 x_0
  \end{pmatrix}
\end{gather}
となる. 
点$x$を引数にとり, 勾配ベクトル$\nabla f(x)$を返すjuliaの関数は以下のコード\ref{code:2}の通り.

\lstinputlisting[
  caption={$\nabla f(x)$を出力する関数},
  label={code:2},
  linerange={7-14}
]{question18/question18.jl}


\subsection{xを入力として$\nabla^2 f(x)$を出力する関数の作成}
\label{sec:q17-3}
ヘッセ行列$\nabla^2 f(x)$は
\begin{gather}
  \nabla^2 f(x) =
  \begin{pmatrix}
    2 + e^{x_0} & -2 \\
    -2 & 12 x_1^2 + 2
  \end{pmatrix}
\end{gather}
となる. 
点$x$を引数にとり, ヘッセ行列$\nabla^2 f(x)$を返すjuliaの関数は以下のコード\ref{code:3}の通り.

\lstinputlisting[
  caption={$\nabla^2 f(x)$を出力する関数},
  label={code:3},
  linerange={16-23}
]{question18/question18.jl}


\section{課題18 バックトラック法を取り入れた2次元の最適化問題}
\label{sec:q18}
課題17の式\eqref{eq:q17}の最適化問題の解を数値的に求める.
その際にアルゴリズム\ref{algo:bisec}, \ref{algo:newton}のステップサイズ$t_n$を決める方法として, 
以下のアルゴリズム\ref{algo:bt}のバックトラック法を採用する. 

\begin{algoitembox}{バックトラック法}
  \label{algo:bt}
  \begin{description}
    \item[step 0:]パラメータ$\xi \in (0,1),\ \rho \in (0,1)$及び初期ステップサイズ$\bar{t} > 0$を選ぶ. 
    \item[step 1:]次式を満たすまで$t \gets \rho t$とする. 
    \begin{gather}
      f(x^n + t d^n) \geq f(x^n) + \xi t \langle d^n, \nabla f(x^n) \rangle
    \end{gather}
    \item[step 2:]上式を満たせば, $t_n = t$とする. 
  \end{description}
\end{algoitembox}

今回は$\xi = 10^{-4},\ \rho = 0.5,\ \bar{t} = 1$と設定した. 


\subsection{バックトラック法を取り入れた最急降下法による最適化問題の解の探索}
\label{sec:q18-1}
アルゴリズム\ref{algo:sdm}の最急降下法のステップサイズの決め方にアルゴリズム\ref{algo:bt}を用いて
式\eqref{eq:q17}の最適化問題の解を数値的に求めた. 
その最適解, 最適値, 反復回数は以下の表\ref{tab:q18-1}の通り. 

\begin{table}[ht]
  \centering
  \caption{最急降下法による式\eqref:q17}の数値計算結果}
  \label{tab:q18-1}
  \begin{tabular}[ht]{|c|c|}
    \hline
    最適解 & $(-0.73345, -0.49333)^\top$ \\
    最適値 & $3.5971$ \\
    反復回数 & 31回 \\
    \hline
  \end{tabular}
\end{table}


\subsection{バックトラック法を取り入れたニュートン法による最適化問題の解の探索}
\label{sec:q18-2}
アルゴリズム\ref{algo:newton}の最急降下法のステップサイズの決め方にアルゴリズム\ref{algo:bt}を用いて
式\eqref{eq:q17}の最適化問題の解を数値的に求めた. 
その最適解, 最適値, 反復回数は以下の表\ref{tab:q18-2}の通り. 
最適解及び最適値は, 2つの手法の間で有効数字9桁の範囲で一致していた. 

\begin{table}[ht]
  \centering
  \caption{ニュートン法による式\eqref:q17}の数値計算結果}
  \label{tab:q18-2}
  \begin{tabular}[ht]{|c|c|}
    \hline
    最適解 & $(-0.73345, -0.49333)^\top$ \\
    最適値 & $3.5971$ \\
    反復回数 & 6回 \\
    \hline
  \end{tabular}
\end{table}


\section{課題19 ニュートン法の修正}
\label{sec:q19}
アルゴリズム\ref{algo:newton}のニュートン法はヘッセ行列$\nabla^2 f(x_k)$が
一様かつ正定値であるという仮定の下で実行しているが, 
実際には一様正定値でない場合も多々ある. 
その場合には, $\nabla^2 f(x_k)$の最小固有値(負の固有値の中で絶対値が最も大きいもの)を$\lambda_{min}$としたときに, 
$\nabla^2 f(x_K)$の代わりに$\nabla^2 f(x_K) + \lambda_{min} + \alpha$
($\alpha$は$10^{-1}$や$10^{-2}$などの定数)を用いて, ヘッセ行列$\nabla^2 f(x_k)$を正定値になるように補正している. \\
\ref{sec:q18}章のニュートン法にこの修正を加えて, 他のアルゴリズム及び条件は同じにして最適化問題
\begin{alignat}{2}
  \label{eq:q19}
  &\text{minimize} \quad && f(x) = \sum_{i=0}^{2} f_i(x)^2 \\
  &\text{subject to} \quad  && x \in \mathbb{R}^2 \\
\end{alignat}
ただし,$f_i(x) = y_i - [x]_0 (1 - [x]_1^{i+1}) \ (i = 0,1,2)$と定義し, 
$y_0 = 1.5,\ y_1 = 2.25,\ y_2 = 2.625$とする. \\
数値計算で求めた最適解及び各手法の反復回数は, 以下の表\ref{tab:q19}の通り. 
最適解及び最適値は, 2つの手法の間で有効数字7桁の範囲で一致していた. 

\begin{table}[ht]
  \centering
  \caption{式\eqref:q19}の数値計算結果}
  \label{tab:q19}
  \begin{tabular}[ht]{|c|c|}
    \hline
    最適解 & $(3.0000, 0.50000)^\top$ \\
    最急降下法の反復回数 & 791回 \\
    ニュートン法の反復回数 & 5回 \\
    \hline
  \end{tabular}
\end{table}



\clearpage
\section{追加課題 制約条件付き凸二次計画問題}
\label{sec:add}
目的関数が二次で制約条件が線形である、二次計画問題
\begin{alignat}{2}
  \label{eq:add}
  &\text{minimize} \quad && \frac{1}{2} x^\top Q x + c^\top x \\
  &\text{subject to} \quad  && A x \leq b
\end{alignat}
を考える. ただし$Q \in \mathbb{R}^{n \times n}$は対称行列, $c \in \mathbb{R}^n$, 
$A \in \mathbb{R}^{m \times n}$, $b \in \mathbb{R}^m$である. 

\subsection{二次計画問題の例}
\label{sec:add-1}
データを2種類に分類する機械学習は, 
データが全て正しい側に配置されているという線形の制約条件のもとで, 
重みベクトル$w = (w_1, w_2, ..., w_n)^\top$の大きさの二乗
$\| w \|^2 = w_1^2 + w_2^2 + ,,,+ w_n^2$を最小化する問題である. 
この問題は次のように定式化できる.
\begin{alignat}{2}
  \label{eq:svm}
  &\text{minimize} \quad  &&  \frac{1}{2}\| w \|^2 \\
  &\text{subject to} \quad  && y_i(w^\top x_i + b) \geq 1 \ (i = 1,2,...,n)
\end{alignat}
ただし$x_i \in \mathbb{R}^n$は$i$番目のデータの特徴ベクトルで, $y_i$はその正しいクラス, 
$w \in \mathbb{R}^n$は重みベクトルで$b \in \mathbb{R}^n$はバイアスを表す\cite{svm}. \\
(文献と授業で習った記憶を整合したのですが, その場合はどう書けばいいでしょうか)

\subsection{目的関数が凸であることの重要性}
\label{sec:add-2}
二次計画問題\eqref{eq:add}の行列$Q$の固有値が全て非負のとき, その目的関数を凸関数といい, 
二次計画問題\eqref{eq:add}を凸二次計画問題という. 
凸計画問題に対しては, 一般の最適化問題と違って
\begin{enumerate}[label=(\alph*), ref=(\alph*)]
    \item 局所最小解が大域最小解になる \label{item:a}
    \item 最適性の条件を満たす点が大域最小解になる \label{item:b}
    \item 双対問題が元の凸計画問題と同じ最適値をもつ \label{item:c}
\end{enumerate}
という性質がある. 
特に\ref{item:a}, \ref{item:b}について, 
局所最小解や最適性の条件を満たす点を求めることは大域最小解を求めることよりも簡単なため, 
凸二次計画問題の解法は一般の二次計画問題よりも簡単になる. 
また\ref{item:c}について, 双対問題は元の問題と比べて扱いやすい性質があるので, 
理論的にも実用的にも考えやすくなる. 
逆に一般の二次計画問題には\ref{item:a}-\ref{item:c}の性質がなく, 上記のような問題の簡単化ができないため, 
凸二次計画問題と比べて難しくなる\cite{nonlinear}. 


\subsection{凸二次計画問題の解法}
\label{sec:add-3}
凸二次計画問題の解法である双対法は以下の通り. 
双対法は繰り返しのインデックスを$k = 0,1,...$とし, 
$J$最適解となる添字集合$J^k$と解$x^k$を逐次生成していくアルゴリズムである. 

\begin{algoitembox}{双対法}
  \label{algo:dual}
  \begin{description}
    \item[step 0:]$J^0 = \emptyset$とし, 初期点$x^0$を制約条件を全く考慮しない場合の最小解とし, 
    それに対応するラグランジュ乗数を$\bm{\lambda}_J^0$とし, 各$\lambda_j = 0, k = 0$とする. 

    \item[step 1:]$J^k$最適解$x^k$が全ての不等式制約を満たしていれば, $x^k$を解として終了する. 
    そうでなければ, $x^k$が満たしていない制約条件を1つ選び, その添え字を$s$とする. 
    \item[step 2:]\begin{align}
      H_J &= Q^{-1}(I - A_J A_J^+) \text{とし} \\
      z^k &= - H_J a^s \\
      r^k &= - A_J^+ a^s \\
    \end{align} 
    により探索方向$z^k$と$r^k$を定める. ただし$a^s$は行列$A$の第$s$行を表す行ベクトルである.

    \item[step 3:]もし$z^k = 0$ならば$t_k^{(P)} = \infty$とおき, そうでなければ
    \begin{gather}
      \lambda_s = \frac{b_s - (a^s)^\top x^k}{-(a^s)^\top H_J a^s}
    \end{gather}
    により$\lambda_s$を定める. 
    ただし$\hat{x}$は添え字$s$に対する不等式制約を満たしていない$J \cup \lbrace s \rbrace$準最適解である. \\
    もし$r^k \geq 0$ならば$t_k^{(D)} = \infty$とおき, そうでなければ
    \begin{align}
      \hat{\lambda}_J &= -(A_J^\top Q^{-1} A_J)^{-1} b_J - A_J^+ c \\
      t_k^{(D)} &= \min \Biggl\{- \frac{\hat{\lambda}_j}{r_j} 
      \Biggm\vert r_j < 0, j \in J^k \rbrace \Biggr\}
    \end{align}
    により$t_k^{(D)}$を定める. $t_k = \min \Bigl\{ \lambda_s, t_k^{(D)} \Bigr\}$とする. 

    \item[step 4:]もし$t_k = \infty$ならばこの凸二次計画問題\eqref{eq:add}には実行可能解が存在しないとして終了する. 
    $t_k^{(D)} = \infty$のときに手順\ref{item:4-a}を実行し, そうでなければ手順\ref{item:4-b}を実行する. 
    \begin{enumerate}[label=(\alph*), ref=(\alph*)]
    \item 添字集合$J^k$からステップ幅$t_k^{(D)} = - \frac{\lambda_l^k}{r_l}$となる添字$l$を取り除いて
    改めて$J^k$とおき, 
    \begin{gather}
      \lambda_j^k = 
      \begin{cases}
        \lambda_j^k - \frac{r_j \lambda_l^k}{r_l} & j \in J^k \\
        \frac{\lambda_l^k}{r_l} & j = s \\
        0 & \text{それ以外}
      \end{cases}
    \end{gather}
    として, step 2へ戻る.
    \label{item:4-a}

    \item \begin{gather}
      \begin{pmatrix}
        \bar{x} \\
        \bar{\bm{\lambda}}_J \\
        \bar{\lambda}_s        
      \end{pmatrix}
    = \begin{pmatrix}
        x^k \\
        \bm{\lambda}_J^k \\
        \lambda_s^k     
      \end{pmatrix}
    + t_k \begin{pmatrix}
        z^k \\
        r^k \\
        1  
      \end{pmatrix}
    \end{gather}
    とする. もし$\lambda_s < t_k^{(D)}$ならば手順\ref{item:4-b-1}を実行し, 
    そうでなければ手順\ref{item:4-b-2}を実行する.
    \begin{enumerate}[label=(b-\arabic*), ref=(b-\arabic*)]
      \item $(x^{k+1}, \bm{\lambda}_J^{k+1}, \lambda_s^{k+1}) = (\bar{x}, \bar{\bm{\lambda}}_J, \bar{\lambda}_s), 
      J^{k+1} = J^k \cup \lbrace s \rbrace$とおき, $k = k+1$としてstep 2へ戻る.
      \label{item:4-b-1}

      \item $(x^k, \bm{\lambda}_J^k, \bm{\lambda_s^k}) = (\bar{x}, \bar{\bm{\lambda}}_J, \bar{\lambda}_s)$とし,
      添字集合$J^k$からステップ幅$t_k^{(D)} = - \frac{\lambda_l^k}{r_l}$となる添字$l$を取り除いて, 
      改めて$J^k$とおいてstep 2へ戻る. 
      \label{item:4-b-2}
    \end{enumerate}
    \label{item:4-b}
\end{enumerate}
  \end{description}
\end{algoitembox}

\subsection{凸二次計画問題の求解}
\label{sec:add-4}
凸二次計画問題\eqref{eq:add}として, 次のパラメータをもつ問題を考える. 
\begin{align}
  Q &=\begin{pmatrix}
    2 & 0 \\
    0 & 1
  \end{pmatrix}
  , \quad c = -\begin{pmatrix}
    1 \\
    1
  \end{pmatrix} \\
  A &= \begin{pmatrix}
    1 & 1
  \end{pmatrix}
  , \quad b = 0
\end{align}
数値解$x^*$を求めるためにjuliaの最適化を行う標準的なパッケージJuMPと
二次計画問題に適したソルバーIpoptを用いた. \\
その結果最適解$x^*$は
\begin{gather}
  x^* = \begin{pmatrix}
    0 \\
    0
  \end{pmatrix}
\end{gather}
と求まった. 


\clearpage
\section{実行環境}
\label{sec:env}
本課題の実行環境は以下の表\ref{tab:env}の通り.

\begin{table}[ht]
  \centering
  \caption{実行環境}
  \label{tab:env}
  \begin{tabular}[ht]{c c}
    \hline
    OS & Ubuntu 22.04.5 LTS \\
    CPU & 13th Gen Intel(R) Core(TM) i7-1360P \\
    メモリ & 8GB \\
    使用言語 & julia 1.12.1 \\
    \hline
  \end{tabular}
\end{table}

\section{参考文献}
\label{sec:ref}
\begin{thebibliography}{9} % {9} は文献数が1桁(1〜9個)の場合のインデント幅。10〜99個なら {99}。

\bibitem{svm}
Kenta Nakamura, 『機械学習の定番「サポートベクターマシン(SVM)」を高校生でもわかるよう解説』, 
https://qiita.com/c60evaporator/items/8864f7c1384a3c6e9bd9, 
最終更新日:2022/03/24, 閲覧日2025/12/28.

\bibitem{nonlinear}
山下信雄, 『非線形計画法』, 朝倉書店, 2015, pp.36-37,91-102.


\end{thebibliography}

数値計算のためのコードで誤っている箇所の修正, 及びより良い書き方の勉強のため, 
Gemini 3.0 Proを使用した.


\end{document}