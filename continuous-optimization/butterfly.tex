\documentclass[uplatex,a4j]{jsarticle}
\usepackage[dvipdfmx]{graphicx}
\usepackage{amsmath,ascmac}
\usepackage{bm}
\usepackage{algorithm,algorithmic}
\usepackage{listings}
\usepackage{subcaption}
\usepackage{amsfonts}
\usepackage{amssymb}
\usepackage{amsmath}
\lstset{%
  language={C},
  basicstyle={\small},%
  identifierstyle={\small},%
  commentstyle={\small\itshape},%
  keywordstyle={\small\bfseries},%
  ndkeywordstyle={\small},%
  stringstyle={\small\ttfamily},
  frame={tb},
  breaklines=true,
  columns=[l]{fullflexible},%
  numbers=left,%
  xrightmargin=0zw,%
  xleftmargin=3zw,%
  numberstyle={\scriptsize},%
  stepnumber=1,
  numbersep=1zw,%
  lineskip=-0.5ex%
}
\renewcommand{\lstlistingname}{コード}
\renewcommand{\lstlistlistingname}{コード目次}
\renewcommand{\thesubsection}{(\arabic{subsection})}
\renewcommand{\thesection}{\arabic{section}}


\begin{document}

\begin{titlepage}
    \centering
    \vfill
    
    {\Huge 数理工学実験\par}
    \vspace{1cm}
    
    {\Large 連続最適化レポート\par}
    
    \vfill
    
    {\Large
    所属: 工学部情報学科数理工学コース2年\par
    学籍番号: 1029-36-1263\par
    氏名: 天野 塁\par
    }
    
    \vfill
    
    {\Large 提出日: \today \par}
    
    \vfill
\end{titlepage}

\section{課題15 二分法及びニュートン法の実践}
\label{sec:q15}
次の関数$f: \mathbb{R} \to \mathbb{R}$
\begin{gather}
  f(x) := x^3 + 2x^2 - 5x - 6
\end{gather}
の零点を数値的に求める. 

\subsection{グラフの描画}
\label{sec:q15-1}
関数$y = f(x)$を$-10 \leq x \leq 10, \ -10 \leq y \leq 10$の範囲で描画すると, 図\ref{fig:q15-1}のようになる. 
グラフの概形を見ると, 零点は$x = -3, -1, 2$付近に存在していると読み取れる. 

\begin{figure}[ht]
  \centering
  \includegraphics[width=0.8\textwidth]{question15/q15.png}
  \caption{関数$y = f(x)$の概形}
  \label{fig:q15-1}
\end{figure}


\subsection{二分法による零点の探索}
\label{sec:q15-2}
二分法を用いて関数$f(x)$の零点を数値的に求める. 二分法のアルゴリズムは以下の通り. 

\begin{itembox}[l]{二分法}
  \begin{description}
    \item[step 0:]$f(a_0)<0, f(b_0) \geq 0$を満たす初期点$a_0, b_0$を選び, 
    終了条件$\epsilon > 0$を決める. 
    \item[step 1:] 中間点 $c_n := (a_n + b_n)/2$ を求める. 
    求めた$c_n$が終了条件$|f(c_n)| \leq \epsilon$を満たせば, $c_n$を零点として終了する. 
    \item[step 2:]$f(c_n) < 0$であれば$a_{n+1} = c_n,\ b_{b+1} = b_n$とし, 
    $f(c_n) \geq 0$であれば$a_{n+1} = a_n,\ b_{b+1} = c_n$として, step 1へ戻る. 
  \end{description}
\end{itembox}

図\ref{fig:q15-1}のグラフの概形を元に, 初期値の組$a_0, b_0$の間にちょうど1つだけ零点が含まれるように初期値を設定した. 
二分法で求めた関数$f(x)$の零点とその初期値の組は以下の表\ref{tab:q15-2}の通り. 

\begin{table}[ht]
  \centering
  \caption{二分法で求めた零点}
  \label{tab:q15-2}
  \begin{tabular}[ht]{c c}
    \hline
    零点 & 初期値の組 \\
    \hline
    $-3.0000$ & $(a_0, b_0) = (-3.3, -2.7)$ \\
    $-1.0000$ & $(a_0, b_0) = (-0.7, -1.3)$ \\
    $2.0000$ & $(a_0, b_0) = (1.7, 2.3)$ \\
    \hline
  \end{tabular}
\end{table}


\subsection{ニュートン法による零点の探索}
\label{sec:q15-3}
ニュートン法を用いて関数$f(x)$の零点を数値的に求める. ニュートン法のアルゴリズムは以下の通り. 

\begin{itembox}[l]{ニュートン法}
  \begin{description}
    \item[step 0:]初期点$x_0$を選び, 終了条件$\epsilon > 0$を決める. 
    \item[step 1:]$\Delta x_n = \frac{-f(x_n)}{f'(x_n)}$とし, 
    $x_{n+1} = x_n + \Delta x_n$とする. 
    \item[step 2:]求めた$x_{n+1}$が終了条件$|f(c_N)| \leq \epsilon$を満たせば, $x_{n+1}$を零点として終了する. 
    満たさなければstep 1に戻る. 
  \end{description}
\end{itembox}

図\ref{fig:q15-1}のグラフの概形を元に, 零点の近くに初期値を設定した. 
二分法で求めた関数$f(x)$の零点とその初期値の組は以下の表\ref{tab:q15-3}の通り. 

\begin{table}[ht]
  \centering
  \caption{ニュートン法で求めた零点}
  \label{tab:q15-3}
  \begin{tabular}[ht]{c c}
    \hline
    零点 & 初期値 \\
    \hline
    $-3.0000$ & $x_0 = -3.3$ \\
    $-1.0000$ & $x_0 = -1.3$ \\
    $2.0000$ & $x_0 = 2.3$ \\
    \hline
  \end{tabular}
\end{table}




\clearpage
\section{実行環境}
\label{sec:env}
本課題の実行環境は以下の表\ref{tab:env}の通り.

\begin{table}[ht]
  \centering
  \caption{実行環境}
  \label{tab:env}
  \begin{tabular}[ht]{c c}
    \hline
    OS & Ubuntu 22.04.5 LTS \\
    CPU & 13th Gen Intel(R) Core(TM) i7-1360P \\
    メモリ & 8GB \\
    使用言語 & julia 1.12.1 \\
    \hline
  \end{tabular}
\end{table}

\section{参考文献}
\label{sec:ref}
数値計算のためのコードで誤っている箇所の修正, 及びより良い書き方の勉強のため, 
Gemini 3.0 Proを使用した.


\end{document}