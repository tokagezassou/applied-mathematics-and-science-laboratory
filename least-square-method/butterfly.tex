\documentclass[uplatex,a4j]{jsarticle}
\usepackage[dvipdfmx]{graphicx}
\usepackage{amsmath,ascmac}
\usepackage{bm}
\usepackage{algorithm,algorithmic}
\usepackage{listings}
\usepackage{subcaption}
\usepackage{amsfonts}
\lstset{%
  language={C},
  basicstyle={\small},%
  identifierstyle={\small},%
  commentstyle={\small\itshape},%
  keywordstyle={\small\bfseries},%
  ndkeywordstyle={\small},%
  stringstyle={\small\ttfamily},
  frame={tb},
  breaklines=true,
  columns=[l]{fullflexible},%
  numbers=left,%
  xrightmargin=0zw,%
  xleftmargin=3zw,%
  numberstyle={\scriptsize},%
  stepnumber=1,
  numbersep=1zw,%
  lineskip=-0.5ex%
}
\renewcommand{\lstlistingname}{コード}
\renewcommand{\lstlistlistingname}{コード目次}
\renewcommand{\thesubsection}{(\arabic{subsection})}
\renewcommand{\thesection}{\arabic{section}}

\begin{document}

\begin{titlepage}
    \centering
    \vfill
    
    {\Huge 数理工学実験\par}
    \vspace{1cm}
    
    {\Large 最小二乗法レポート\par}
    
    \vfill
    
    {\Large
    所属: 工学部情報学科数理工学コース2年\par
    学籍番号: 1029-36-1263\par
    氏名: 天野 塁\par
    }
    
    \vfill
    
    {\Large 提出日: \today \par}
    
    \vfill
\end{titlepage}

\section{課題1 最小二乗法を用いた重回帰問題の解答}
\label{sec:q1}

\subsection{全データを用いたパラメータの推定}
\label{sec:q1-1}
10000個全てのデータを用いて推定した最小二乗誤差推定量$\hat{\theta}_N$及び推定共分散行列$\hat{V}_N$は
式\ref{val:q1-1-1}, \ref{val:q1-1-2}の通り. 
推定共分散行列の各成分のオーダーが$10^{-4}$から$10^{-7}$だったので, 十分収束していると言える.

\begin{align}
  \hat{\theta}_N &= 
  \begin{pmatrix}
    1.5066 \\
    1.9977
  \end{pmatrix}
  \label{val:q1-1-1}
  \\
  \hat{V}_N &= 
  \begin{pmatrix}
    9.8665 & -0.040817 \\
    -0.040817 & 10.052
  \end{pmatrix} \times 10^{-5}
  \label{val:q1-1-2}
\end{align}

\subsection{最小二乗誤差推定量$\hat{\theta}_N$の収束確認}
\label{sec:q1-2}
用いるデータ数$N$を$N = 2,4,8,...,2^{13} = 8192$と増やしていったときの, 
最小二乗誤差推定量$\hat{\theta}_N$の値をプロットしたグラフは図\ref{fig:q1-2}の通り. 
グラフには$\hat{\theta}_N$の各成分を有効数字2桁で表した$y=1.5, y = 2.0$の直線を付している. 
それらの直線とプロットされた点を比較すると, 最小二乗誤差推定量$\hat{\theta}_N$は十分収束しているといえる. 

\begin{figure}[ht]
  \centering
  \includegraphics[width=0.8\textwidth]{question1/q1.png}
  \caption{最小二乗誤差推定量の収束}
  \label{fig:q1-2}
\end{figure}

\subsection{全データを用いた決定変数の計算}
\label{sec:q1-3}
10000個全てのデータを用いて推定した決定変数$C$は
式\ref{val:q1-3}の通り. 
値が$0$よりも$1$に大幅に近いため, 式\ref{val:q1-1-1}の推定値は信頼できるといえる. 

\begin{align}
  C = 0.86297
  \label{val:q1-3}
\end{align}


\section{課題2 最小二乗法多項式回帰の解答}
\label{sec:q2}

\subsection{全データを用いたパラメータの推定}
\label{sec:q2-1}
10000個全てのデータを用いて推定した最小二乗誤差推定量$\hat{\theta}_N$及び推定共分散行列$\hat{V}_N$は
式\ref{val:q2-1-1}, \ref{val:q2-1-2}の通り. 
推定共分散行列の各成分のオーダーが$10^{-3}$から$10^{-8}$だったので, 十分収束していると言える.

\begin{align}
  \hat{\theta}_N &= 
  \begin{pmatrix}
    -0.50902 \\
    1.9759 \\
    0.19774 \\
    -0.098667
  \end{pmatrix}
  \label{val:q2-1-1}
  \\
  \hat{V}_N &= 
  \begin{pmatrix}
    202.34 & -1.1865 & -000013.483 & 0.039712 \\
    -1.1865 & 67.530 & -0.018985 & -3.7947 \\
    -13.483 & -0.018985 & 1.6039 & 0.0063518 \\
    0.039712 & -3.79472 & 0.0063518 & 0.25287
  \end{pmatrix} \times 10^{-5}
  \label{val:q2-1-2}
\end{align}

\subsection{最小二乗誤差推定量$\hat{\theta}_N$の収束確認}
\label{sec:q2-2}
用いるデータ数$N$を$N = 4,8,...,2^{13} = 8192$と増やしていったときの, 
最小二乗誤差推定量$\hat{\theta}_N$の値をプロットしたグラフは図\ref{fig:q2-2}の通り. 
グラフには$\hat{\theta}_N$の各成分を有効数字2桁で表した
$y=-0.50, y = 2.0, y = 0.20, y = -0.099$の直線を付している. 
それらの直線とプロットされた点を比較すると, 最小二乗誤差推定量$\hat{\theta}_N$の第1,3,4成分は十分収束しているといえる. 
ただし第2成分は変化が緩やかにはなっているが, 他の成分と比較して収束しているとはいえない. 

\begin{figure}[ht]
  \centering
  \includegraphics[width=0.8\textwidth]{question2/q2.png}
  \caption{最小二乗誤差推定量の収束}
  \label{fig:q2-2}
\end{figure}

\subsection{全データを用いた決定変数の計算}
\label{sec:q2-3}
10000個全てのデータを用いて推定した決定変数$C$は
式\ref{val:q2-3}の通り. 
値が$0$にも$1$にも近くないため, 式\ref{val:q2-1-1}の推定値はあまり信頼できないといえる. 

\begin{align}
  C = 0.46185
  \label{val:q2-3}
\end{align}


\newpage
\section{課題3 分散$\infty$の観測誤差の場合の最小二乗法の実践}
\label{sec:q3}

\subsection{全データを用いたパラメータの推定}
\label{sec:q3-1}
10000個全てのデータを用いて推定した最小二乗誤差推定量$\hat{\theta}_N$は
式\ref{val:q3-1}の通り. 

\begin{align}
  \hat{\theta}_N &= 
  \begin{pmatrix}
    2.3731 \\
    1.5373
  \end{pmatrix}
  \label{val:q3-1}
\end{align}

\subsection{最小二乗誤差推定量$\hat{\theta}_N$の収束確認}
\label{sec:q3-2}
用いるデータ数$N$を$N = 2,4,8,...,2^{13} = 8192$と増やしていったときの, 
最小二乗誤差推定量$\hat{\theta}_N$の値をプロットしたグラフは図\ref{fig:q3-2}の通り. 
グラフには$\hat{\theta}_N$の各成分を有効数字2桁で表した
$y=2.4, y = 1.5$の直線を付している. 
それらの直線とプロットされた点を比較すると, 最小二乗誤差推定量$\hat{\theta}_N$はほとんど収束していないといえる. 

\begin{figure}[ht]
  \centering
  \includegraphics[width=0.8\textwidth]{question3/q3.png}
  \caption{最小二乗誤差推定量の収束}
  \label{fig:q3-2}
\end{figure}


\newpage
\section{課題4 入力を狭めた場合の最小二乗法の実践}
\label{sec:q4}

1000個のデータを用いて推定した最小二乗誤差推定量$\hat{\theta}_N$及び推定共分散行列$\hat{V}_N$は
式\ref{val:q4-1}, \ref{val:q4-2}の通り. 
推定共分散行列の各成分のオーダーが$10^{-1}$から$10^{2}$だったので, 
課題2の式\ref{val:q2-1-2}と比較してほとんど収束していないといえる. \\
課題2と違う点は2つあり, 1つはデータ数が少ないこと, もう1つは入力値$x_i$の範囲が$[0,1]$に限定されていることである. 
データ数が少ないことで推測に使う方程式が少なくなっており, 
また入力値の範囲が絞られていることで方程式の係数の種類が少なくなり, 推測に有効な方程式の数がさらに少なくなっている. 
したがって課題2と比較して収束が弱く, 値の信頼性も低くなっている. 
つまり信頼性の高い推定値を得るためには, データの数を増やして, 
かつ入力値と出力値の範囲を広く設定するべきだといえる.

\begin{align}
  \hat{\theta}_N &= 
  \begin{pmatrix}
    -0.57772 \\
    2.0968 \\
    -0.39506 \\
    0.54912
  \end{pmatrix}
  \label{val:q4-1}
  \\
  \hat{V}_N &= 
  \begin{pmatrix}
    0.15324 & -1.1488 & 2.2964 & -1.3389 \\
    -1.1488 & 11.459 & -25.752 & 16.010 \\
    2.2964 & -25.752 & 61.720 & -39.966 \\
    -1.3389 & 16.010 & -39.966 & 26.617
  \end{pmatrix}
  \label{val:q4-2}
\end{align}


\section{課題5 重み付き最小二乗法の実践}
\label{sec:q5}

1000個のデータを用いて重みを考慮せずに推定した最小二乗誤差推定量$\hat{\theta}_N$及び推定共分散行列$\hat{V}_N$は
式\ref{val:q5-s1}, \ref{val:q5-s2}の通り. 
また重みを考慮して推定した最小二乗誤差推定量$\hat{\theta}_{Nw}$及び推定共分散行列$\hat{V}_{Nw}$は
式\ref{val:q5-w1}, \ref{val:q5-w2}の通りで, 用いるデータ数$N$を$N = 2,4,8,...,2^9 = 512$と増やしていったときの, 
最小二乗誤差推定量$\hat{\theta}_{Nw}$の値をプロットしたグラフは図\ref{fig:q5}の通り.
グラフには$\hat{\theta}_N$の各成分を有効数字2桁で表した
$y=-0.50, y = 2.0, y = 0.20, y = -0.099$の直線を付している. \\
式\ref{val:q5-s1}, \ref{val:q5-w1}より推定値は近い値になっているが, 
式\ref{val:q5-s2}, \ref{val:q5-w2}より推定共分散行列の各成分の値には$10^3$のオーダーの差があり, 
重みを考慮して求めた推定値の方が信頼できるといえる. 
また図\ref{fig:q5}より, 最小二乗誤差推定量$\hat{\theta}_{Nw}$は十分収束しているといえる. 

\begin{align}
  \hat{\theta}_N &= 
  \begin{pmatrix}
    2.9946 \\
    -2.0690
  \end{pmatrix}
  \label{val:q5-s1}
  \\
  \hat{V}_N &= 
  \begin{pmatrix}
    351.45 & -125.16 \\
    -125.16 & 108.60
  \end{pmatrix} \times 10^{-4}
  \label{val:q5-s2}
  \\
  \hat{\theta}_{Nw} &= 
  \begin{pmatrix}
    2.9391 \\
    -1.9865
  \end{pmatrix}
  \label{val:q5-w1}
  \\
  \hat{V}_{Nw} &= 
  \begin{pmatrix}
    14.774 & -5.0005 \\
    -5.0005 & 5.1312
  \end{pmatrix} \times 10^{-4}
  \label{val:q5-w2}
\end{align}

\begin{figure}[ht]
  \centering
  \includegraphics[width=0.8\textwidth]{question5/q5.png}
  \caption{最小二乗誤差推定量の収束}
  \label{fig:q5}
\end{figure}




\end{document}