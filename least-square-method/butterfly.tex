\documentclass[uplatex,a4j]{jsarticle}
\usepackage[dvipdfmx]{graphicx}
\usepackage{amsmath,ascmac}
\usepackage{bm}
\usepackage{algorithm,algorithmic}
\usepackage{listings}
\usepackage{subcaption}
\usepackage{amsfonts}
\lstset{%
  language={C},
  basicstyle={\small},%
  identifierstyle={\small},%
  commentstyle={\small\itshape},%
  keywordstyle={\small\bfseries},%
  ndkeywordstyle={\small},%
  stringstyle={\small\ttfamily},
  frame={tb},
  breaklines=true,
  columns=[l]{fullflexible},%
  numbers=left,%
  xrightmargin=0zw,%
  xleftmargin=3zw,%
  numberstyle={\scriptsize},%
  stepnumber=1,
  numbersep=1zw,%
  lineskip=-0.5ex%
}
\renewcommand{\lstlistingname}{コード}
\renewcommand{\lstlistlistingname}{コード目次}
\renewcommand{\thesubsection}{(\arabic{subsection})}
\renewcommand{\thesection}{\arabic{section}}

\begin{document}

\begin{titlepage}
    \centering
    \vfill
    
    {\Huge 数理工学実験\par}
    \vspace{1cm}
    
    {\Large 最小二乗法レポート\par}
    
    \vfill
    
    {\Large
    所属: 工学部情報学科数理工学コース2年\par
    学籍番号: 1029-36-1263\par
    氏名: 天野 塁\par
    }
    
    \vfill
    
    {\Large 提出日: \today \par}
    
    \vfill
\end{titlepage}

\section{課題1 重回帰問題}
\label{sec:q1}

\subsection{全データを用いたパラメータの推定}
\label{sec:q1-1}
10000個全てのデータを用いて推定した最小二乗誤差推定量$\hat{\theta}_N$及び推定共分散行列$\hat{V}_N$は
式\ref{val:q1-1-1}, \ref{val:q1-1-2}の通り. 
推定共分散行列の各成分のオーダーが$10^{-4}$ - $10^{-7}$だったので, 十分収束していると言える.

\begin{align}
  \hat{\theta}_N &= 
  \begin{pmatrix}
    1.5066 \\
    1.9977
  \end{pmatrix}
  \label{val:q1-1-1}
  \\
  \hat{V}_N &= 
  \begin{pmatrix}
    986.65 & -4.0817 \\
    -4.0817 & 1005.2
  \end{pmatrix} \times 10^{-7}
  \label{val:q1-1-2}
\end{align}

\subsection{最小二乗誤差推定量$\hat{\theta}_N$の収束確認}
\label{sec:q1-2}
用いるデータ数$N$を$N = 2,4,8,...,2^{13} = 8192$と増やしていったときの, 
最小二乗誤差推定量$\hat{\theta}_N$の値をプロットしたグラフは図\ref{fig:q1-2}の通り. 
グラフには$\hat{\theta}_N$の各成分を有効数字2桁で表した$y=1.5$と$y = 2.0$の直線を付している. 
その直線とプロットされた点を比較することによっても, 最小二乗誤差推定量$\hat{\theta}_N$の収束を確認できる. 

\begin{figure}[ht]
  \centering
  \includegraphics[width=0.8\textwidth]{question1/q1.png}
  \caption{最小二乗誤差推定量の収束}
  \label{fig:q1-2}
\end{figure}

\subsection{全データを用いた決定変数の計算}
\label{sec:q1-3}
10000個全てのデータを用いて推定した決定変数$C$は
式\ref{val:q1-3}の通り. 
値が$0$よりも$1$に大幅に近いため, 式\ref{val:q1-1-1}の推定値は信頼できるといえる. 

\begin{align}
  C = 0.86297
  \label{val:q1-3}
\end{align}


\end{document}