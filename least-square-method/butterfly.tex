\documentclass[uplatex,a4j]{jsarticle}
\usepackage[dvipdfmx]{graphicx}
\usepackage{amsmath,ascmac}
\usepackage{bm}
\usepackage{algorithm,algorithmic}
\usepackage{listings}
\usepackage{subcaption}
\usepackage{amsfonts}
\lstset{%
  language={C},
  basicstyle={\small},%
  identifierstyle={\small},%
  commentstyle={\small\itshape},%
  keywordstyle={\small\bfseries},%
  ndkeywordstyle={\small},%
  stringstyle={\small\ttfamily},
  frame={tb},
  breaklines=true,
  columns=[l]{fullflexible},%
  numbers=left,%
  xrightmargin=0zw,%
  xleftmargin=3zw,%
  numberstyle={\scriptsize},%
  stepnumber=1,
  numbersep=1zw,%
  lineskip=-0.5ex%
}
\renewcommand{\lstlistingname}{コード}
\renewcommand{\lstlistlistingname}{コード目次}
\renewcommand{\thesubsection}{(\arabic{subsection})}
\renewcommand{\thesection}{\arabic{section}}

\begin{document}

\begin{titlepage}
    \centering
    \vfill
    
    {\Huge 数理工学実験\par}
    \vspace{1cm}
    
    {\Large 最小二乗法レポート\par}
    
    \vfill
    
    {\Large
    所属: 工学部情報学科数理工学コース2年\par
    学籍番号: 1029-36-1263\par
    氏名: 天野 塁\par
    }
    
    \vfill
    
    {\Large 提出日: \today \par}
    
    \vfill
\end{titlepage}

\section{課題1 最小二乗法を用いた重回帰問題の解答}
\label{sec:q1}

\subsection{全データを用いたパラメータの推定}
\label{sec:q1-1}
10000個全てのデータを用いて推定した最小二乗誤差推定量$\hat{\theta}_N$及び推定共分散行列$\hat{V}_N$は
式\ref{val:q1-1-1}, \ref{val:q1-1-2}の通り. 
推定共分散行列の各成分のオーダーが$10^{-4}$から$10^{-7}$だったので, 十分収束していると言える.

\begin{align}
  \hat{\theta}_N &= 
  \begin{pmatrix}
    1.5066 \\
    1.9977
  \end{pmatrix}
  \label{val:q1-1-1}
  \\
  \hat{V}_N &= 
  \begin{pmatrix}
    9.8665 & -0.040817 \\
    -0.040817 & 10.052
  \end{pmatrix} \times 10^{-5}
  \label{val:q1-1-2}
\end{align}

\subsection{最小二乗誤差推定量$\hat{\theta}_N$の収束確認}
\label{sec:q1-2}
用いるデータ数$N$を$N = 2,4,8,...,2^{13} = 8192$と増やしていったときの, 
最小二乗誤差推定量$\hat{\theta}_N$の値をプロットしたグラフは図\ref{fig:q1-2}の通り. 
グラフには$\hat{\theta}_N$の各成分を有効数字2桁で表した$y=1.5, y = 2.0$の直線を付している. 
それらの直線とプロットされた点を比較すると, 最小二乗誤差推定量$\hat{\theta}_N$は十分収束しているといえる. 

\begin{figure}[ht]
  \centering
  \includegraphics[width=0.8\textwidth]{question1/q1.png}
  \caption{最小二乗誤差推定量の収束}
  \label{fig:q1-2}
\end{figure}

\subsection{全データを用いた決定変数の計算}
\label{sec:q1-3}
10000個全てのデータを用いて推定した決定変数$C$は
式\ref{val:q1-3}の通り. 
値が$0$よりも$1$に大幅に近いため, 式\ref{val:q1-1-1}の推定値は信頼できるといえる. 

\begin{align}
  C = 0.86297
  \label{val:q1-3}
\end{align}


\section{課題2 最小二乗法多項式回帰の解答}
\label{sec:q2}

\subsection{全データを用いたパラメータの推定}
\label{sec:q2-1}
10000個全てのデータを用いて推定した最小二乗誤差推定量$\hat{\theta}_N$及び推定共分散行列$\hat{V}_N$は
式\ref{val:q2-1-1}, \ref{val:q2-1-2}の通り. 
推定共分散行列の各成分のオーダーが$10^{-3}$から$10^{-8}$だったので, 十分収束していると言える.

\begin{align}
  \hat{\theta}_N &= 
  \begin{pmatrix}
    -0.50902 \\
    1.9759 \\
    0.19774 \\
    -0.098667
  \end{pmatrix}
  \label{val:q2-1-1}
  \\
  \hat{V}_N &= 
  \begin{pmatrix}
    202.34 & -1.1865 & -000013.483 & 0.039712 \\
    -1.1865 & 67.530 & -0.018985 & -3.7947 \\
    -13.483 & -0.018985 & 1.6039 & 0.0063518 \\
    0.039712 & -3.79472 & 0.0063518 & 0.25287
  \end{pmatrix} \times 10^{-5}
  \label{val:q2-1-2}
\end{align}

\subsection{最小二乗誤差推定量$\hat{\theta}_N$の収束確認}
\label{sec:q2-2}
用いるデータ数$N$を$N = 4,8,...,2^{13} = 8192$と増やしていったときの, 
最小二乗誤差推定量$\hat{\theta}_N$の値をプロットしたグラフは図\ref{fig:q2-2}の通り. 
グラフには$\hat{\theta}_N$の各成分を有効数字2桁で表した
$y=-0.50, y = 2.0, y = 0.20, y = -0.099$の直線を付している. 
それらの直線とプロットされた点を比較すると, 最小二乗誤差推定量$\hat{\theta}_N$の第1,3,4成分は十分収束しているといえる. 
ただし第2成分は変化が緩やかにはなっているが, 他の成分と比較して収束しているとはいえない. 

\begin{figure}[ht]
  \centering
  \includegraphics[width=0.8\textwidth]{question2/q2.png}
  \caption{最小二乗誤差推定量の収束}
  \label{fig:q2-2}
\end{figure}

\subsection{全データを用いた決定変数の計算}
\label{sec:q2-3}
10000個全てのデータを用いて推定した決定変数$C$は
式\ref{val:q2-3}の通り. 
値が$0$にも$1$にも近くないため, 式\ref{val:q2-1-1}の推定値はあまり信頼できないといえる. 

\begin{align}
  C = 0.46185
  \label{val:q2-3}
\end{align}


\newpage
\section{課題3 分散$\infty$の観測誤差の場合の最小二乗法の実践}
\label{sec:q3}

\subsection{全データを用いたパラメータの推定}
\label{sec:q3-1}
10000個全てのデータを用いて推定した最小二乗誤差推定量$\hat{\theta}_N$は
式\ref{val:q3-1}の通り. 

\begin{align}
  \hat{\theta}_N &= 
  \begin{pmatrix}
    2.3731 \\
    1.5373
  \end{pmatrix}
  \label{val:q3-1}
\end{align}

\subsection{最小二乗誤差推定量$\hat{\theta}_N$の収束確認}
\label{sec:q3-2}
用いるデータ数$N$を$N = 2,4,8,...,2^{13} = 8192$と増やしていったときの, 
最小二乗誤差推定量$\hat{\theta}_N$の値をプロットしたグラフは図\ref{fig:q3-2}の通り. 
グラフには$\hat{\theta}_N$の各成分を有効数字2桁で表した
$y=2.4, y = 1.5$の直線を付している. 
それらの直線とプロットされた点を比較すると, 最小二乗誤差推定量$\hat{\theta}_N$はほとんど収束していないといえる. 

\begin{figure}[ht]
  \centering
  \includegraphics[width=0.8\textwidth]{question3/q3.png}
  \caption{最小二乗誤差推定量の収束}
  \label{fig:q3-2}
\end{figure}


\newpage
\section{課題4 入力を狭めた場合の最小二乗法の実践}
\label{sec:q4}

1000個のデータを用いて推定した最小二乗誤差推定量$\hat{\theta}_N$及び推定共分散行列$\hat{V}_N$は
式\ref{val:q4-1}, \ref{val:q4-2}の通り. 
推定共分散行列の各成分のオーダーが$10^{-1}$から$10^{2}$だったので, 
課題2の式\ref{val:q2-1-2}と比較してほとんど収束していないといえる. \\
課題2と違う点は2つあり, 1つはデータ数が少ないこと, もう1つは入力値$x_i$の範囲が$[0,1]$に限定されていることである. 
データ数が少ないことで推測に使う方程式が少なくなっており, 
また入力値の範囲が絞られていることで方程式の係数の種類が少なくなり, 推測に有効な方程式の数がさらに少なくなっている. 
したがって課題2と比較して収束が弱く, 値の信頼性も低くなっている. 
つまり信頼性の高い推定値を得るためには, データの数を増やして, 
かつ入力値と出力値の範囲を広く設定するべきだといえる.

\begin{align}
  \hat{\theta}_N &= 
  \begin{pmatrix}
    -0.57772 \\
    2.0968 \\
    -0.39506 \\
    0.54912
  \end{pmatrix}
  \label{val:q4-1}
  \\
  \hat{V}_N &= 
  \begin{pmatrix}
    0.15324 & -1.1488 & 2.2964 & -1.3389 \\
    -1.1488 & 11.459 & -25.752 & 16.010 \\
    2.2964 & -25.752 & 61.720 & -39.966 \\
    -1.3389 & 16.010 & -39.966 & 26.617
  \end{pmatrix}
  \label{val:q4-2}
\end{align}


\section{課題5 重み付き最小二乗法の実践}
\label{sec:q5}

1000個のデータを用いて重みを考慮せずに推定した最小二乗誤差推定量$\hat{\theta}_N$及び推定共分散行列$\hat{V}_N$は
式\ref{val:q5-s1}, \ref{val:q5-s2}の通り. 
また重みを考慮して推定した最小二乗誤差推定量$\hat{\theta}_{Nw}$及び推定共分散行列$\hat{V}_{Nw}$は
式\ref{val:q5-w1}, \ref{val:q5-w2}の通りで, 用いるデータ数$N$を$N = 2,4,8,...,2^9 = 512$と増やしていったときの, 
最小二乗誤差推定量$\hat{\theta}_{Nw}$の値をプロットしたグラフは図\ref{fig:q5}の通り.
グラフには$\hat{\theta}_N$の各成分を有効数字2桁で表した
$y=-0.50, y = 2.0, y = 0.20, y = -0.099$の直線を付している. \\
式\ref{val:q5-s1}, \ref{val:q5-w1}より推定値は近い値になっているが, 
式\ref{val:q5-s2}, \ref{val:q5-w2}より推定共分散行列の各成分の値には$10^3$のオーダーの差があり, 
重みを考慮して求めた推定値の方が信頼できるといえる. 
また図\ref{fig:q5}より, 最小二乗誤差推定量$\hat{\theta}_{Nw}$は十分収束しているといえる. 

\begin{align}
  \hat{\theta}_N &= 
  \begin{pmatrix}
    2.9946 \\
    -2.0690
  \end{pmatrix}
  \label{val:q5-s1}
  \\
  \hat{V}_N &= 
  \begin{pmatrix}
    351.45 & -125.16 \\
    -125.16 & 108.60
  \end{pmatrix} \times 10^{-4}
  \label{val:q5-s2}
  \\
  \hat{\theta}_{Nw} &= 
  \begin{pmatrix}
    2.9391 \\
    -1.9865
  \end{pmatrix}
  \label{val:q5-w1}
  \\
  \hat{V}_{Nw} &= 
  \begin{pmatrix}
    14.774 & -5.0005 \\
    -5.0005 & 5.1312
  \end{pmatrix} \times 10^{-4}
  \label{val:q5-w2}
\end{align}

\begin{figure}[ht]
  \centering
  \includegraphics[width=0.8\textwidth]{question5/q5.png}
  \caption{最小二乗誤差推定量の収束}
  \label{fig:q5}
\end{figure}


\section{課題6 異なるデータセットの入力を持つ重み付き最小二乗法の実践}
\label{sec:q6}

1000個のデータを用いて重みを考慮せずに推定した最小二乗誤差推定量$\hat{\theta}_N$は
式\ref{val:q6-s}の通りで, 用いるデータ数$N$を$N = 2,4,8,...,2^9 = 512$と増やしていったときの, 
最小二乗誤差推定量$\hat{\theta}_{N}$の値をプロットしたグラフは図\ref{fig:q6-s}の通り.
グラフには$\hat{\theta}_N$の各成分を有効数字2桁で表した
$y = 3.1, y = -2.1$の直線を付している. \\
また重みを考慮して推定した最小二乗誤差推定量$\hat{\theta}_{Nw}$は
式\ref{val:q6-w}の通りで, 用いるデータ数$N$を$N = 2,4,8,...,2^9 = 512$と増やしていったときの, 
最小二乗誤差推定量$\hat{\theta}_{Nw}$の値をプロットしたグラフは図\ref{fig:q6-w}の通り.
グラフには$\hat{\theta}_N$の各成分を有効数字2桁で表した
$y = 3.0, y = -2.0$の直線を付している. \\
式\ref{val:q6-s}, \ref{val:q6-w}より推定値は近い値になっている. 
また図\ref{fig:q6-w}では十分な収束が確認できるが, 図\ref{fig:q6-s}では特に第1成分で収束が確認できない. 
したがって重みを考慮して計算した推定値の方が信頼性が高いといえる. 

\begin{align}
  \hat{\theta}_N &= 
  \begin{pmatrix}
    3.1884 \\
    -2.0922
  \end{pmatrix}
  \label{val:q6-s}
  \\
  \hat{\theta}_{Nw} &= 
  \begin{pmatrix}
    2.9942 \\
    -2.0147
  \end{pmatrix}
  \label{val:q6-w}
\end{align}

\begin{figure}[htbp]
  \centering

  \begin{subfigure}{0.45\textwidth}
    \centering
    \includegraphics[width=\linewidth]{question6/q6_standard.png}
    \caption{重み無し}
    \label{fig:q6-s}
  \end{subfigure}
  \hfill
  \begin{subfigure}{0.45\textwidth}
    \centering
    \includegraphics[width=\linewidth]{question6/q6_weighted.png}
    \caption{重み付き}
    \label{fig:q6-w}
  \end{subfigure}
  
  \caption{最小二乗誤差推定量の収束}
  \label{fig:q6}
\end{figure}


\section{課題7 入力のデータセットを分割した場合の重み付き最小二乗法の実践}
\label{sec:q7}

10000組のデータを6000組と4000組に分割したとき, 
最初の6000組のデータを用いて推定した最小二乗誤差推定量$\hat{\theta}_{N=6000}$は式\ref{val:q7-1}の通りで, 
残りの4000組のデータを用いて推定した最小二乗誤差推定量$\hat{\theta}_{N=4000}$は式\ref{val:q7-2}の通り. 
またその2つを合成して求めた最小二乗誤差推定量$\hat{\theta}_{Nsynth}$は式\ref{val:q7-3}の通りで, 
全てのデータを用いて推定した最小二乗誤差推定量$\hat{\theta}_{Nwhole}$は式\ref{val:q7-4}の通り. \\
式\ref{val:q7-3}, \ref{val:q7-4}より, 有効数字5桁の範囲では推定値は確かに一致している. 

\begin{align}
  \hat{\theta}_{N=6000} &= 
  \begin{pmatrix}
    -0.0038794 \\
    3.0107 \\
    -1.98943
  \end{pmatrix}
  \label{val:q7-1}
  \\
  \hat{\theta}_{N=4000} &= 
  \begin{pmatrix}
    -0.025376 \\
    3.0349 \\
    -1.9777
  \end{pmatrix}
  \label{val:q7-2}
  \\
  \hat{\theta}_{Nsynth} &= 
  \begin{pmatrix}
    -0.012495 \\
    3.0204 \\
    -1.9849
  \end{pmatrix}
  \label{val:q7-3}
  \\
  \hat{\theta}_{Nwhole} &= 
  \begin{pmatrix}
    -0.012495 \\
    3.0204 \\
    -1.9849
  \end{pmatrix}
  \label{val:q7-4}
\end{align}


\section{課題8 異なる入力のデータセットを含む場合の重み付き最小二乗法の実践}
\label{sec:q8}

10000組のデータを6000組と4000組に分割したとき, 
最初の6000組のデータを用いて推定した最小二乗誤差推定量$\hat{\theta}_{N=6000}$
及び偶然誤差の分散$\hat{V}_{N=6000}$は式\ref{val:q8-f1}, \ref{val:q8-f2}の通りで, 
残りの4000組のデータを用いて推定した最小二乗誤差推定量$\hat{\theta}_{N=4000}$
及び偶然誤差の分散$\hat{V}_{N=4000}$は式\ref{val:q8-l1}, \ref{val:q8-l2}の通りで, 
またその2つを合成して求めた最小二乗誤差推定量$\hat{\theta}_{Nsynth}$は式\ref{val:q8-s}の通りで, 
全てのデータを用いて推定した最小二乗誤差推定量$\hat{\theta}_{Nwhole}$は式\ref{val:q8-w}の通り. \\
式\ref{val:q8-f2}, \ref{val:q8-l2}より, 後半4000組の分散は前半6000組と比較して極端に小さいため, 
合成して求めた推定値\ref{val:q8-s}は後半4000組のデータのみを用いて求めた推定値\ref{val:q8-l1}と
ほぼ同じ値になっている. 

\begin{align}
  \hat{\theta}_{N=6000} &= 
  \begin{pmatrix}
    0.0070768 \\
    3.28054 \\
    -2.1909
  \end{pmatrix}
  \label{val:q8-f1}
  \\
  \hat{V}_{N=6000} &= 96.863
  \label{val:q8-f2}
  \\
  \hat{\theta}_{N=4000} &= 
  \begin{pmatrix}
    0.098483 \\
    3.1004 \\
    -2.0910
  \end{pmatrix}
  \label{val:q8-l1}
  \\
  \hat{V}_{N=4000} &= 0.010304
  \label{val:q8-l2}
  \\
  \hat{\theta}_{Nsynth} &= 
  \begin{pmatrix}
    0.098469 \\
    3.1004 \\
    -2.0910
  \end{pmatrix}
  \label{val:q8-s}
  \\
  \hat{\theta}_{Nwhole} &= 
  \begin{pmatrix}
    0.043732 \\
    3.2087 \\
    -2.1512
  \end{pmatrix}
  \label{val:q8-w}
\end{align}

データセットごとに入力を分割して扱う手法と, 全データを重みを考慮せずに一気に扱う手法を比較するため, 
データセット$(x_i,y_i)$の重み$z_i$を考慮した, 式\ref{doc}で表される決定変数$C_z$を用いた. 
\begin{align}
  C_z = \frac{\sum_{i=1}^{N} z_i \| \phi_i \hat{\theta}_N - \bar{y}_z \|}
  {\sum_{i=1}^{N} z_i \| y_i - \bar{y}_z \|}
  \label{doc} \\
  \text{ただし$\bar{y}_z$は$y_i$の重み付き平均で} \\
  \bar{y}_z = \frac{\sum_{i=1}^{N} z_i y_i}{\sum_{i=1}^{N} z_i}
\end{align}

式\ref{doc}の決定変数の, データセットごとに入力を分割して扱う場合の値$C_z^{div}$及び
全データを重みを考慮せずに一気に扱う場合の値$C_z^{whole}$は, 
それぞれ式\ref{val:q8-c1}, \ref{val:q8-c2}の通り. 
式\ref{val:q8-c1}より分割して扱った場合は決定変数の値が極めて$1$に近いのに対し, 
式\ref{val:q8-c2}より全てのデータを等しく扱った場合は決定変数の値が極めて$0$に近い. 
したがって推定の精度は, データセットごとに分割して重みを考慮して推定した方が圧倒的に高くなるといえる. \\
一方計算にかかった時間は分割して扱った場合は$2.1379$秒なのに対し, 
全てのデータを等しく扱った場合は$0.010149$秒であり, 
計算時間は全てのデータを等しく扱った方が圧倒的に少なくなっている. \\
これらのことから, 異なるデータセットの分散の差が大きい場合には, 精度の高いデータを得るために, 
データセットごとにデータを分割してそれぞれの分散を元に重みを考慮する方が良いといえる. 
一方異なるデータセットの分散の差があまり大きくない場合には, 重みの差も大きくならないので, 
計算時間を短縮するために全てのデータを等しく扱う方が良いといえる. 
一概にどちらの手法が優れているというものではないため, 
データの特徴を踏まえてどちらの手法を採用するか検討する必要がある. 

\begin{align}
  C_z^{div} &= 0.98413
  \label{val:q8-c1}
  \\
  C_z^{whole} &= 0.028670
  \label{val:q8-c2}
\end{align}


\section{課題9 最小二乗法を用いたシステム同定}
\label{sec:q9}

\subsection{一定の力を与えた場合}
\label{sec:q9-1}
$F_k = 1$という一定の力を与えた場合のパラメータ$\theta$の推定値$\hat{\theta}_N$及びそのときの$M, D, K$の値は
式\ref{val:q9-1-1}から\ref{val:q9-1-4}の通り. 
$M, D, K$の値は真値と大きく異なるため, この力の定め方は不適切といえる. 

\begin{align}
  \hat{\theta}_N &= 
  \begin{pmatrix}
    1.9951 \\
    -0.99527 \\
    -0.0051824
  \end{pmatrix}
  \label{val:q9-1-1}
  \\
  M &= -0.019296 \ \text{(真値は$2$)}
  \label{val:q9-1-2}
  \\
  D &= -0.0094211 \ \text{(真値は$1$)}
  \label{val:q9-1-3}
  \\
  K &= -0.029401 \ \text{(真値は$3$)}
  \label{val:q9-1-4}
\end{align}

\subsection{$\sin$関数の力を与えた場合}
\label{sec:q9-2}
$F_k = \sin{\frac{\pi k}{5}}$という力を与えた場合のパラメータ$\theta$の推定値$\hat{\theta}_N$
及びそのときの$M, D, K$の値は式\ref{val:q9-2-1}から\ref{val:q9-2-4}の通り. 
$M, D, K$の値は真値と大きく異なるため, この力の定め方は不適切といえる. 

\begin{align}
  \hat{\theta}_N &= 
  \begin{pmatrix}
    1.9946 \\
    -0.99480 \\
    0.015189
  \end{pmatrix}
  \label{val:q9-2-1}
  \\
  M &= 0.0065836 \ \text{(真値は$2$)}
  \label{val:q9-2-2}
  \\
  D &= 0.0035340 \ \text{(真値は$1$)}
  \label{val:q9-2-3}
  \\
  K &= 0.011283 \ \text{(真値は$3$)}
  \label{val:q9-2-4}
\end{align}

\subsection{パラメータ推定に適する力の考察}
\label{sec:q9-3}
\ref{sec:q9-1}, \ref{sec:q9-2}節で扱った力の問題点は, 大きく次の2つがある. 
1つ目は力が小さすぎて誤差の影響を受けやすいこと, 
2つ目は力が一定もしくは周期が短く, 同じ係数の方程式が多数できることで, 
推定に有効な方程式の数が少ないことだ. \\

まずは1つめの問題点を解決するため, 力を$F_k = 10^6$と極端に大きい値にした. 
この場合のパラメータ$\theta$の推定値$\hat{\theta}_N$
及びそのときの$M, D, K$の値は式\ref{val:q9-3-1}から\ref{val:q9-3-4}の通り. 
$M, D, K$の値は真値とほぼ等しく, 精度は極めて高い. 
しかし実際の実験の場でこのように極端に大きい力を与えることは不可能である. 

\begin{align}
  \hat{\theta}_N &= 
  \begin{pmatrix}
    1.9950 \\
    -0.99513 \\
    5.0066 \times 10^{-5}
  \end{pmatrix}
  \label{val:q9-3-1}
  \\
  M &= 1.9974 \ \text{(真値は$2$)}
  \label{val:q9-3-2}
  \\
  D &= 1.0019 \ \text{(真値は$1$)}
  \label{val:q9-3-3}
  \\
  K &= 2.9993 \ \text{(真値は$3$)}
  \label{val:q9-3-4}
\end{align}

次に力を大きくしすぎることなく有効な方程式の数を増やすため, 力を$F_k = 500 \sin{1.7 k}$とした. 
$\sin$の引数に$\pi$を入れないことで$F_k$が全く同じ値を取ることを防ぎ, 
$1.7$を掛けることで$\pi \approx 3$の倍数に近い値をなるべく取らないようにした. 
この場合のパラメータ$\theta$の推定値$\hat{\theta}_N$
及びそのときの$M, D, K$の値は式\ref{val:q9-4-1}から\ref{val:q9-4-4}の通り. 
$M, D, K$の値は真値に近く, 精度は高いといえる. 
さらに力$F_k$の振幅も\ref{sec:q9-3}節より大幅に小さいため, 十分実用的である. 
実際の実験の場では, 力$F_k$の振幅と精度がトレードオフの関係になっている. 

\begin{align}
  \hat{\theta}_N &= 
  \begin{pmatrix}
    1.9962 \\
    -0.99633 \\
    4.4636 \times 10^{-5}
  \end{pmatrix}
  \label{val:q9-4-1}
  \\
  M &= 2.2403 \ \text{(真値は$2$)}
  \label{val:q9-4-2}
  \\
  D &= 0.85523 \ \text{(真値は$1$)}
  \label{val:q9-4-3}
  \\
  K &= 3.2791 \ \text{(真値は$3$)}
  \label{val:q9-4-4}
\end{align}

\newpage
\section{課題10 非定常時系列データの推定}
\label{sec:q10}
$y_k = \sin{0.00001 k} + w_K$で表される信号に対して$\theta_k := sin{0.00001 k}$が未知であるとき, 
$\gamma = 0.99$の忘却係数付きの逐次二乗法で推定した各時刻ごとの$\theta_k$の値は図\ref{fig:q10}の通り. 
誤差による上下のブレはあるものの, 平均的には正しく推定できているといえる. 

\begin{figure}[ht]
  \centering
  \includegraphics[width=0.8\textwidth]{question10/q10.png}
  \caption{非定常時系列データの推定値と真値}
  \label{fig:q10}
\end{figure}


\newpage
\section{課題11 Kalmanフィルタを用いたダイナミクスの推定}
\label{sec:q11}
ダイナミクス
\begin{gather}
  \theta_k = 0.9 \theta_{k^1} + v_k
\end{gather}
及び観測方程式
\begin{gather}
  y_k = 2 \theta_k + w_k
\end{gather}
に対して, 時刻$k _ 1,...,100$における$\theta_k$の値及びその推定値$\hat{\theta}_k$は図\ref{fig:q11}の通り. 
推定値は真値と近い分布となっており, 精度の高い推定といえる. 

\begin{figure}[ht]
  \centering
  \includegraphics[width=0.8\textwidth]{question11/q11.png}
  \caption{Kalmanフィルタを用いた推定値と真値}
  \label{fig:q11}
\end{figure}


\newpage
\section{課題12 Kalmanスムーザを用いた初期分散の推定}
\label{sec:q12}
ダイナミクス
\begin{gather}
  \theta_k = 0.9 \theta_{k^1} + v_k
\end{gather}
及び観測方程式
\begin{gather}
  y_k = 2 \theta_k + w_k
\end{gather}
に対して, \ref{sec:q11}節で求めた分布を元に初期値及び初期分散を推定する. 
初期値$\theta_0$, 初期分散$V_0$, 初期分散$V_k^s$に対する推定誤差分散の比$\frac{V_0}{V_k^s}$は, 
それぞれ式\ref{val:q11-1}, \ref{val:q11-2}, \ref{val:q11-3}の通り. 
式\ref{val:q11-3}より推定誤差分散が初期分散と比較して$0.42842$になっているので, 
初期分散の推定には成功しているといえる. 

\begin{align}
  \theta_0 &= 4.1870
  \label{val:q11-1}
  \\
  V_0 &= 0.85685
  \label{val:q11-2}
  \\
  \frac{V_0}{V_k^s} &= 0.42842
  \label{val:q11-3}
\end{align}



\end{document}