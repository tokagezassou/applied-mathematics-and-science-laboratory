\documentclass[uplatex,a4j]{jsarticle}
\usepackage[dvipdfmx]{graphicx}
\usepackage{amsmath,ascmac}
\usepackage{bm}
\usepackage{algorithm,algorithmic}
\usepackage{listings}
\usepackage{subcaption}
\usepackage{amsfonts}
\lstset{%
  language={C},
  basicstyle={\small},%
  identifierstyle={\small},%
  commentstyle={\small\itshape},%
  keywordstyle={\small\bfseries},%
  ndkeywordstyle={\small},%
  stringstyle={\small\ttfamily},
  frame={tb},
  breaklines=true,
  columns=[l]{fullflexible},%
  numbers=left,%
  xrightmargin=0zw,%
  xleftmargin=3zw,%
  numberstyle={\scriptsize},%
  stepnumber=1,
  numbersep=1zw,%
  lineskip=-0.5ex%
}
\renewcommand{\lstlistingname}{コード}
\renewcommand{\lstlistlistingname}{コード目次}
\renewcommand{\thesubsection}{(\arabic{subsection})}
\renewcommand{\thesection}{\arabic{section}}

\begin{document}

\begin{titlepage}
    \centering
    \vfill
    
    {\Huge 数理工学実験\par}
    \vspace{1cm}
    
    {\Large 最小二乗法レポート\par}
    
    \vfill
    
    {\Large
    所属: 工学部情報学科数理工学コース2年\par
    学籍番号: 1029-36-1263\par
    氏名: 天野 塁\par
    }
    
    \vfill
    
    {\Large 提出日: \today \par}
    
    \vfill
\end{titlepage}

\section{課題1 重回帰問題}
\label{sec:q1}

\subsection{全データを用いたパラメータの推定}
\label{sec:q1-1}
10000個全てのデータを用いて推定した最小二乗誤差推定量$\hat{\theta}_N$及び推定共分散行列$\hat{V}_N$は
式\ref{val:q1-1-1}, \ref{val:q1-1-2}の通り. 
推定共分散行列の各成分のオーダーが$10^{-4}$から$10^{-7}$だったので, 十分収束していると言える.

\begin{align}
  \hat{\theta}_N &= 
  \begin{pmatrix}
    1.5066 \\
    1.9977
  \end{pmatrix}
  \label{val:q1-1-1}
  \\
  \hat{V}_N &= 
  \begin{pmatrix}
    9.8665 & -0.040817 \\
    -0.040817 & 10.052
  \end{pmatrix} \times 10^{-5}
  \label{val:q1-1-2}
\end{align}

\subsection{最小二乗誤差推定量$\hat{\theta}_N$の収束確認}
\label{sec:q1-2}
用いるデータ数$N$を$N = 2,4,8,...,2^{13} = 8192$と増やしていったときの, 
最小二乗誤差推定量$\hat{\theta}_N$の値をプロットしたグラフは図\ref{fig:q1-2}の通り. 
グラフには$\hat{\theta}_N$の各成分を有効数字2桁で表した$y=1.5, y = 2.0$の直線を付している. 
それらの直線とプロットされた点を比較すると, 最小二乗誤差推定量$\hat{\theta}_N$は十分収束しているといえる. 

\begin{figure}[ht]
  \centering
  \includegraphics[width=0.8\textwidth]{question1/q1.png}
  \caption{最小二乗誤差推定量の収束}
  \label{fig:q1-2}
\end{figure}

\subsection{全データを用いた決定変数の計算}
\label{sec:q1-3}
10000個全てのデータを用いて推定した決定変数$C$は
式\ref{val:q1-3}の通り. 
値が$0$よりも$1$に大幅に近いため, 式\ref{val:q1-1-1}の推定値は信頼できるといえる. 

\begin{align}
  C = 0.86297
  \label{val:q1-3}
\end{align}


\section{課題2 多項式回帰}
\label{sec:q2}

\subsection{全データを用いたパラメータの推定}
\label{sec:q2-1}
10000個全てのデータを用いて推定した最小二乗誤差推定量$\hat{\theta}_N$及び推定共分散行列$\hat{V}_N$は
式\ref{val:q2-1-1}, \ref{val:q2-1-2}の通り. 
推定共分散行列の各成分のオーダーが$10^{-3}$から$10^{-8}$だったので, 十分収束していると言える.

\begin{align}
  \hat{\theta}_N &= 
  \begin{pmatrix}
    -0.50902 \\
    1.9759 \\
    0.19774 \\
    -0.098667
  \end{pmatrix}
  \label{val:q2-1-1}
  \\
  \hat{V}_N &= 
  \begin{pmatrix}
    202.34 & -1.1865 & -000013.483 & 0.039712 \\
    -1.1865 & 67.530 & -0.018985 & -3.7947 \\
    -13.483 & -0.018985 & 1.6039 & 0.0063518 \\
    0.039712 & -3.79472 & 0.0063518 & 0.25287
  \end{pmatrix} \times 10^{-5}
  \label{val:q2-1-2}
\end{align}

\subsection{最小二乗誤差推定量$\hat{\theta}_N$の収束確認}
\label{sec:q2-2}
用いるデータ数$N$を$N = 4,8,...,2^{13} = 8192$と増やしていったときの, 
最小二乗誤差推定量$\hat{\theta}_N$の値をプロットしたグラフは図\ref{fig:q2-2}の通り. 
グラフには$\hat{\theta}_N$の各成分を有効数字2桁で表した
$y=-0.50, y = 2.0, y = 0.20, y = -0.099$の直線を付している. 
それらの直線とプロットされた点を比較すると, 最小二乗誤差推定量$\hat{\theta}_N$の第1,3,4成分は十分収束しているといえる. 
ただし第2成分は変化が緩やかにはなっているが, 他の成分と比較して収束しているとはいえない. 

\begin{figure}[ht]
  \centering
  \includegraphics[width=0.8\textwidth]{question2/q2.png}
  \caption{最小二乗誤差推定量の収束}
  \label{fig:q2-2}
\end{figure}

\subsection{全データを用いた決定変数の計算}
\label{sec:q2-3}
10000個全てのデータを用いて推定した決定変数$C$は
式\ref{val:q2-3}の通り. 
値が$0$にも$1$にも近くないため, 式\ref{val:q2-1-1}の推定値はあまり信頼できないといえる. 

\begin{align}
  C = 0.46185
  \label{val:q2-3}
\end{align}


\newpage
\section{課題3 分散$\infty$の観測誤差の場合}
\label{sec:q3}

\subsection{全データを用いたパラメータの推定}
\label{sec:q3-1}
10000個全てのデータを用いて推定した最小二乗誤差推定量$\hat{\theta}_N$は
式\ref{val:q3-1}の通り. 

\begin{align}
  \hat{\theta}_N &= 
  \begin{pmatrix}
    2.3731 \\
    1.5373
  \end{pmatrix}
  \label{val:q3-1}
\end{align}

\subsection{最小二乗誤差推定量$\hat{\theta}_N$の収束確認}
\label{sec:q3-2}
用いるデータ数$N$を$N = 2,4,8,...,2^{13} = 8192$と増やしていったときの, 
最小二乗誤差推定量$\hat{\theta}_N$の値をプロットしたグラフは図\ref{fig:q3-2}の通り. 
グラフには$\hat{\theta}_N$の各成分を有効数字2桁で表した
$y=2.4, y = 1.5$の直線を付している. 
それらの直線とプロットされた点を比較すると, 最小二乗誤差推定量$\hat{\theta}_N$はほとんど収束していないといえる. 

\begin{figure}[ht]
  \centering
  \includegraphics[width=0.8\textwidth]{question2/q2.png}
  \caption{最小二乗誤差推定量の収束}
  \label{fig:q3-2}
\end{figure}


\newpage
\section{課題4 入力を狭めた場合}
\label{sec:q4}

1000個のデータを用いて推定した最小二乗誤差推定量$\hat{\theta}_N$及び推定共分散行列$\hat{V}_N$は
式\ref{val:q4-1}, \ref{val:q4-2}の通り. 
推定共分散行列の各成分のオーダーが$10^{-3}$から$10^{-8}$だったので, 十分収束していると言える.

\begin{align}
  \hat{\theta}_N &= 
  \begin{pmatrix}
    -0.57772 \\
    2.0968 \\
    -0.39506 \\
    0.54912
  \end{pmatrix}
  \label{val:q4-1}
  \\
  \hat{V}_N &= 
  \begin{pmatrix}
    0.15324370924567504 -1.148750956149419 2.296352397230773 -1.3388694636527916; -1.1487509561494356 11.458867152413008 -25.752415167470524 16.010390549216996; 2.2963523972308386 -25.752415167470794 61.720072741152904 -39.96601404328088; -1.3388694636528433 16.01039054921729 -39.96601404328116 26.617391970216755
  \end{pmatrix} \times 10^{-5}
  \label{val:q4-2}
\end{align}






\end{document}