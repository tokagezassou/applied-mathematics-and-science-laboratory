% !TEX program = lualatex
\documentclass[...]{ltjsarticle}
\usepackage{graphicx}
\usepackage{amsmath,ascmac}
\usepackage{bm}
\usepackage{algorithm,algorithmic}
\usepackage{listings}
\usepackage{subcaption}
\usepackage{amsfonts}
\usepackage{amssymb}
\usepackage{amsmath}
\usepackage{enumitem}
\lstdefinelanguage{Julia}%
  {morekeywords={abstract,break,case,catch,const,continue,do,else,elseif,end,export,false,for,function,immutable,import,importall,if,in,macro,module,quote,return,struct,true,try,type,typealias,using,while},%
   sensitive=true,%
   morecomment=[l]{\#},%      % # を \# に修正
   morecomment=[s]{\#=}{=\#},% % # を \# に修正
   morestring=[b]",%
   morestring=[m]'%
  }
\lstset{%
  language={Julia}, % Juliaをメインで使うならここを {Julia} にしてもOKです
  basicstyle={\small\ttfamily}, % 読みやすさのため ttfamily を推奨
  identifierstyle={\small},%
  commentstyle={\small\itshape},%
  keywordstyle={\small\bfseries},%
  ndkeywordstyle={\small},%
  stringstyle={\small\ttfamily},
  frame={tb},
  breaklines=true,
  columns=fullflexible, % [l] を外すとより安定します
  keepspaces=true, % スペースを維持
  numbers=left,%
  xrightmargin=0pt, % 0zw から 0pt に変更
  xleftmargin=3em,  % 3zw から 3em に変更(1zw ≒ 1em です)
  numberstyle={\scriptsize},%
  stepnumber=1,
  numbersep=1em,    % 1zw から 1em に変更
  lineskip=-0.5ex%
}
\renewcommand{\lstlistingname}{コード}
\renewcommand{\lstlistlistingname}{コード目次}
\renewcommand{\thesubsection}{(\arabic{subsection})}
\renewcommand{\thesection}{\arabic{section}}

\begin{document}

\begin{titlepage}
    \centering
    \vfill
    
    {\Huge 数理工学実験\par}
    \vspace{1cm}
    
    {\Large 組合せ最適化レポート\par}
    
    \vfill
    
    {\Large
    所属: 工学部情報学科数理工学コース2年\par
    学籍番号: 1029-36-1263\par
    氏名: 天野 塁\par
    }
    
    \vfill
    
    {\Large 提出日: \today \par}
    
    \vfill
\end{titlepage}


\section{課題1 最短路を求める分枝アルゴリズムの設計}
\label{sec:q1}





\clearpage
\section{実行環境}
\label{sec:env}
本課題の実行環境は表\ref{tab:env}の通りである.

\begin{table}[ht]
  \centering
  \caption{実行環境}
  \label{tab:env}
  \begin{tabular}[ht]{c c}
    \hline
    OS & Ubuntu 22.04.5 LTS \\
    CPU & 13th Gen Intel(R) Core(TM) i7-1360P \\
    メモリ & 8GB \\
    使用言語 & julia 1.12.1 \\
    \hline
  \end{tabular}
\end{table}

\section{参考文献}
\label{sec:ref}
数値計算のためのコードで誤っている箇所の修正, 及びより良い書き方の勉強のため, 
Gemini 3.0 Proを使用した.


\end{document}