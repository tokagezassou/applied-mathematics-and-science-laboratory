% !TEX program = lualatex
\documentclass[...]{ltjsarticle}
\usepackage{graphicx}
\usepackage{amsmath,ascmac}
\usepackage{bm}
\usepackage{algorithm}
\usepackage{listings}
\usepackage{subcaption}
\usepackage{amsfonts}
\usepackage{amssymb}
\usepackage{amsmath}
\usepackage{enumitem}
\usepackage{algpseudocode}
\usepackage{multirow}
\usepackage{graphicx}
\usepackage{makecell}
\usepackage{booktabs}
\lstdefinelanguage{Julia}%
  {morekeywords={abstract,break,case,catch,const,continue,do,else,elseif,end,export,false,for,function,immutable,import,importall,if,in,macro,module,quote,return,struct,true,try,type,typealias,using,while},%
   sensitive=true,%
   morecomment=[l]{\#},%      % # を \# に修正
   morecomment=[s]{\#=}{=\#},% % # を \# に修正
   morestring=[b]",%
   morestring=[m]'%
  }
\lstset{%
  language={Julia}, % Juliaをメインで使うならここを {Julia} にしてもOKです
  basicstyle={\small\ttfamily}, % 読みやすさのため ttfamily を推奨
  identifierstyle={\small},%
  commentstyle={\small\itshape},%
  keywordstyle={\small\bfseries},%
  ndkeywordstyle={\small},%
  stringstyle={\small\ttfamily},
  frame={tb},
  breaklines=true,
  columns=fullflexible, % [l] を外すとより安定します
  keepspaces=true, % スペースを維持
  numbers=left,%
  xrightmargin=0pt, % 0zw から 0pt に変更
  xleftmargin=3em,  % 3zw から 3em に変更(1zw ≒ 1em です)
  numberstyle={\scriptsize},%
  stepnumber=1,
  numbersep=1em,    % 1zw から 1em に変更
  lineskip=-0.5ex%
}
\renewcommand{\lstlistingname}{コード}
\renewcommand{\lstlistlistingname}{コード目次}
\renewcommand{\thesubsection}{(\arabic{subsection})}
\renewcommand{\thesection}{\arabic{section}}

\algrenewcommand\algorithmicif{\textbf{if}}
\algrenewcommand\algorithmicthen{\textbf{then}}
\algrenewcommand\algorithmicelse{\textbf{else}}
\algrenewcommand\algorithmicfor{\textbf{for}}
\algrenewcommand\algorithmicforall{\textbf{for each}}
\algrenewcommand\algorithmicdo{\textbf{do}}
\algrenewcommand\algorithmicwhile{\textbf{while}}
\algrenewcommand\algorithmicprocedure{\textbf{procedure}}
\algrenewcommand\algorithmicfunction{\textbf{function}}
\algnewcommand\algorithmicreturnvalue{\textbf{return}}
\algnewcommand\ReturnValue[1]{\State \algorithmicreturnvalue\ #1}
\algnewcommand\algorithmicoutput{\textbf{output}}
\algnewcommand\Output[1]{\State \algorithmicoutput\ #1}
\algrenewcommand\algorithmicend{\textbf{end}}
\algrenewcommand\algorithmicreturn{\textbf{return}}

\begin{document}

\begin{titlepage}
    \centering
    \vfill
    
    {\Huge 数理工学実験\par}
    \vspace{1cm}
    
    {\Large 熱方程式の差分法レポート\par}
    
    \vfill
    
    {\Large
    所属: 工学部情報学科数理工学コース2年\par
    学籍番号: 1029-36-1263\par
    氏名: 天野 塁\par
    }
    
    \vfill
    
    {\Large 提出日: \today \par}
    
    \vfill
\end{titlepage}


\section{問題1 拡散方程式の数値解}
\label{sec:q1}

拡散方程式
\begin{gather}
  \label{eq:dif}
  \frac{\partial u}{\partial t} (x,t) = \frac{\partial^2 u}{\partial x^2} (x,t)
\end{gather}
をサイズ$L = 10$及び初期条件
\begin{gather}
  u_0(x) = \frac{1}{\sqrt{2 \pi}} e^{-\frac{1}{2} (x-5)^2}
\end{gather}
の下で数値的に解いた. 
数値解法はEuler陽解法
\begin{gather}
  u_j^{n+1} - u_j^{n} \approx 
  \frac{\Delta t}{\Delta x^2} (u_{j-1}^n - 2u_j^n + u_{j+1}^n)
\end{gather}
及びCrank-Nicolson法
\begin{gather}
  u_j^{n+1} - u_j^{n} \approx 
  \frac{1}{2} \frac{\Delta t}{\Delta x^2} (u_{j-1}^{n+1} - 2u_j^{n+1} + u_{j+1}^{n+1}) + 
  \frac{1}{2} \frac{\Delta t}{\Delta x^2} (u_{j-1}^n - 2u_j^n + u_{j+1}^n)
\end{gather}
を用い, 刻み幅は$\Delta t = 0.01,\ \Delta x = 0.5$とした. 
ただし数値解法中の$u$の上付き添字は時間の, 下付き添字は座標のインデックスを表す. 
境界条件はDirichlet境界条件$u_L = u_R = 0$及びNeumann境界条件$J_L = J_R = 0$を用いた. 
それぞれの数値解法及び境界条件の計4通りの組み合わせで, 
$t = 1,2,3,4,5$における$u(x,t)$の値を出力した. 
その結果は図\ref{fig:q1-ed} - \ref{fig:q1-cnn}の通りである. \\
今回のDirichlet境界条件は両端の値が$0$であるという条件のため, 
図\ref{fig:q1-ed},\ref{fig:q1-cnd}において$x = 0, 10$付近の$u(x,t)$の値が
Neumann境界条件を用いた図\ref{fig:q1-en},\ref{fig:q1-cnn}と比較して$0$に近くなっている. 
また数値解法の違いによる$u(x,t)$の値の差はほとんど見られなかった. 

\clearpage
\begin{figure}[htbp]
  \centering

  \begin{minipage}[t]{0.48\textwidth}
    \centering
    \includegraphics[width=\linewidth]{question1/q1-ed.png}
    \captionof{figure}{Euler陽解法・Dirichlet境界条件を用いた拡散方程式の数値解}
    \label{fig:q1-ed}
  \end{minipage}
  \hfill
  \begin{minipage}[t]{0.48\textwidth}
    \centering
    \includegraphics[width=\linewidth]{question1/q1-en.png}
    \captionof{figure}{Euler陽解法・Neumann境界条件を用いた拡散方程式の数値解}
    \label{fig:q1-en}
  \end{minipage}
  
\end{figure}

\begin{figure}[htbp]
  \centering

  \begin{minipage}[t]{0.48\textwidth}
    \centering
    \includegraphics[width=\linewidth]{question1/q1-cnd.png}
    \captionof{figure}{Crank-Nicolson法・Dirichlet境界条件を用いた拡散方程式の数値解}
    \label{fig:q1-cnd}
  \end{minipage}
  \hfill
  \begin{minipage}[t]{0.48\textwidth}
    \centering
    \includegraphics[width=\linewidth]{question1/q1-cnn.png}
    \captionof{figure}{Crank-Nicolson法・Neumann境界条件を用いた拡散方程式の数値解}
    \label{fig:q1-cnn}
  \end{minipage}
  
\end{figure}


\clearpage
\section{問題2 Fisher方程式の数値解}
\label{sec:q2}

式\eqref{eq:dif}の拡散方程式に非線形項$f(u) = u (1 - u)$を加えたFisher方程式
\begin{gather}
  \frac{\partial u}{\partial t} (x,t) = \frac{\partial^2 u}{\partial x^2} (x,t) + u (1 - u)
\end{gather}
をサイズ$L = 200$及び初期条件
\begin{gather}
  u_0(x) = \frac{1}{(1 + e^{bx - 5})^2}
\end{gather}
の下で数値的に解いた. 
数値解法はEuler陽解法
\begin{gather}
  \frac{u_j^{n+1} - u_j^{n}}{\Delta t} \approx 
  \frac{u_{j-1}^n - 2u_j^n + u_{j+1}^n}{\Delta x^2} + f(u_j^n)
\end{gather}
を用い, 刻み幅は$\Delta t = 0.001,\ \Delta x = 0.05$とした. 
ただし数値解法中の$u$の上付き添字は時間の, 下付き添字は座標のインデックスを表す. 
境界条件はDirichlet境界条件$u_L = 1,\ u_R = 0$を用いた. 
パラメータ$b$の値が$b = 0.25, 0.5, 1.0$の3つの場合で, 
$t = 10,20,30,40$における$u(x,t)$の値を出力した. 
その結果は図\ref{fig:q2-quater} - \ref{fig:q2-full}の通りである. \\
$b$及び$t$の値によらず, $u(x,t)$の概形はある$x$を境に$u = 1$から$u = 0$に急激に減少するような形になっており, 
$u(x,t)$が急激に減少する$x$の値は, 時刻$t$が大きくなるほど大きくなっている. 
つまり$u(x,t)$が$1$から$0$へ変化する境界が, 時間経過とともに$x$軸の正の方向に進行しているといえる. 
一方パラメータ$b$の値が大きくなるほど$u(x,t)$が急激に減少する$x$の値は小さくなっており, 
また減少中の傾きの絶対値も大きくなっている. 
つまり$b$は$u=1$から$u=0$への変化の緩急を表すパラメータだといえる. 

\begin{figure}[ht]
  \centering
  \includegraphics[width=0.8\textwidth]{question2/q2-quater.png}
  \caption{$b = 0.25$におけるFisher方程式の数値解}
  \label{fig:q2-quater}
\end{figure}

\begin{figure}[ht]
  \centering
  \includegraphics[width=0.8\textwidth]{question2/q2-half.png}
  \caption{$b = 0.5$におけるFisher方程式の数値解}
  \label{fig:q2-half}
\end{figure}

\begin{figure}[ht]
  \centering
  \includegraphics[width=0.8\textwidth]{question2/q2-full.png}
  \caption{$b = 1.0$におけるFisher方程式の数値解}
  \label{fig:q2-full}
\end{figure}


\clearpage
\section{問題3 1次元調和振動子のSchr\"{o}dinger方程式の数値解}
\label{sec:q3}

Schr\"{o}dinger方程式
\begin{gather}
  i \hbar \frac{\partial \psi}{\partial t} (x,t) = 
  (- \frac{\hbar^2}{2m} \frac{\partial^2}{\partial x^2} + \frac{k}{2}x^2) \psi (x,t)
\end{gather}
を$\psi(x,t) = \psi_R(x,t) + \psi_I(x,t)$として実部と虚部に分けた拡散方程式系
\begin{align}
  \hbar \frac{\partial \psi_R}{\partial t} (x,t) &= 
  (- \frac{\hbar^2}{2m} \frac{\partial^2}{\partial x^2} + \frac{k}{2}x^2) \psi_I (x,t) \\
  - \hbar \frac{\partial \psi_I}{\partial t} (x,t) &= 
  (- \frac{\hbar^2}{2m} \frac{\partial^2}{\partial x^2} + \frac{k}{2}x^2) \psi_R (x,t) \\
\end{align}
を空間$[- \frac{L}{2}, \frac{L}{2}]$, 
パラメータ$\hbar = 1,\ m = 1,\ k = 1$, 及び初期条件
\begin{gather}
  \psi (x,0) = \frac{\sqrt{2}}{\pi^{\frac{1}{4}}} e^{-2 x^2}
\end{gather}
の下で数値的に解いた. 
数値解法はVisscherのスキーム
\begin{align}
  R_j^{n+1} &= R_j^n - 
  \Delta t (-\frac{1}{2} \frac{I_{j-1}^n - 2I_n^n + I_{j+1}^n}{\Delta x^2} + 
  \frac{1}{2} x_j^2 I_j^n) \\
  I_j^{n+1} &= I_j^n + 
  \Delta t (-\frac{1}{2} \frac{R_{j-1}^{n+1} - 2R_n^{n+1} + R_{j+1}^{n+1}}{\Delta x^2} + 
  \frac{1}{2} x_j^2 R_j^{n+1}) \\
  \text{ただし } x_j &= (j - \frac{1}{2}) \Delta x - \frac{L}{2}
\end{align}
を用いた. 
ただし数値解法中の$R$及び$I$の上付き添字は時間の, 下付き添字は座標のインデックスを表す. 
刻み幅は$\Delta t = 0.001,\ \Delta x = 0.05\ (\therefore N = 400)$とし, 
境界条件は周期境界条件
\begin{align}
  &R_0^n = R_N^n, \quad R_{N+1}^n = R_1^n \\
  &I_0^n = I_N^n, \quad I_{N+1}^n = I_1^n
\end{align}
として, 
$t = 1,2,3,4,5,6,7,8$における確率密度
\begin{gather}
  (R_j^n)^2 + I_j^n I_j^{n-1} \ (1 \leq j \leq N)
\end{gather}
の値を出力した. 
その結果は図\ref{fig:q3}の通りである. \\
$t$の値によらず, 確率密度は座標$x_j = 0$でピークとなるように対称に分布している. 
ピークでの値は$t = 3, 6$のときに有意に大きくなっており, 分布も$x_j = 0$の近傍に集中している. 

\begin{figure}[ht]
  \centering
  \includegraphics[width=0.8\textwidth]{question3/q3.png}
  \caption{Schr\"{o}dinger方程式の数値解}
  \label{fig:q3}
\end{figure}


\clearpage
\section{実行環境}
\label{sec:env}
本課題の実行環境は表\ref{tab:env}の通りである.

\begin{table}[ht]
  \centering
  \caption{実行環境}
  \label{tab:env}
  \begin{tabular}[ht]{c c}
    \hline
    OS & Ubuntu 22.04.5 LTS \\
    CPU & 13th Gen Intel(R) Core(TM) i7-1360P \\
    メモリ & 8GB \\
    使用言語 & julia 1.12.1 \\
    \hline
  \end{tabular}
\end{table}

\section{参考文献}
\label{sec:ref}
数値計算のためのコードで誤っている箇所の修正, 及びより良い書き方の勉強のため, 
Gemini 3.0 Proを使用した.


\end{document}